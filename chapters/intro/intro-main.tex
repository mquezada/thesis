\chapter{Introduction}

% WHAT IS SOCIAL MEDIA

The so-called {\em Web 2.0} represents a change of state in how users interact
with the Web.
%
It is mainly defined as a system destined to encourage end-users to publish and
consume content. 
%
One of the main manifestations of this phenomena are the {\em online social
networks}, also called {\em microblogging platforms}.
%
Users online make connections with others based on different criteria, and start
to produce content that might be interesting to other users. 
%
Microblogging services such as Facebook~\cite{facebook}, Twitter~\cite{twitter},
or Sina Weibo~\cite{weibo} are now among the most used platforms to connect with
family, friends, acquaintances, co-workers, or strangers with similar interests.
%
Users interact with each other and produce or share content about their lives,
thoughts, or what is happening around the world, etc.    
%
This collective of information published in these Internet-based applications,
such as microblogging platforms, blogs, wikis, etc., is what is called {\em
social media}~\cite{kaplan2010users}.

%%

% CONTENT IN SM

Content in social media is multimodal. 
%
Twitter, for example, encourages users to publish short texts (initially limited
to 140 characters, now 280), but it recently started to incite users to share
more photographs~\cite{brown_2019}.
%  
And with the proliferation of smartphones and internet-connected devices, more
of this information is also geo-tagged and in ``real-time''. 
%
For instance, Foursquare~\cite{foursquare} is an application for smartphone
users for tagging and commenting on locations around the world, which serves as
a repository of data about businesses and points of interest.
%
Whether via text, images, videos, sounds or hyperlinks, social media lowered the
entry barriers to content producers and made it easy for consumers to access to
a myriad of different pieces of information.


%%


% UTILITY, APPLICATIONS OF SM

The influence of social media on society can not be denied, as it facilitates
the communication between people and speeds-up the difussion of information
online. 
%
For instance, it is believed that the revolutionary wave of protests and
uprisings in the Arab states (known as the Arab Spring which began in 2010) was
highly influenced by social media as a means of organizing and facilitating
communication~\cite{howard2011opening}. 
%
Also, it has permitted many applications to proliferate in emergency management
and detection, such as earthquake alert systems using
Twitter~\cite{Sakaki2010,Sakaki:Tweet:2013,Sarmiento:2018:DDE:3201064.3201077,Mendoza2019}.
%
The usefulness of quantiative analysis of events through time is undeniable, and
social media offers a window to see and capture information about those events,
how they develop, and how the world interprets them.



% NEWS CONSUMPTION AND PRODUCTION


One of the main usages of social media platforms is the consumption and
generation of event-related content. 
%
According to a recent 2018 study~\cite{pewresearch}, about two thirds of U.S.
adults get their news on social media.
%
The most used platforms for news are Facebook, Youtube, and Twitter,
while over 70\% of Twitter users surveyed use that platform to do so.
%
Furthermore, almost every news outlet has a presence in social media
in order to attract readers and viewers.
%
In this way, users comment on the news events, reacting to them according to a
myriad of factors, and many of these characteristics are present in one way or
another in social media.
%
Therefore, we see social media as a medium that reflects an important 
part of what society thinks about what is happening the world.




% PROBLEMS

However, the popularity of social media is not without issues. 
%
{\em Information overload} refers to the problem of being unable to manage or to
make decisions based on data, due to the high volume of information available
and the limited capabilities of the person who is dealing with it. 
%
Humans have limited cognitive processing capacities, and when they are
overloaded with information, their quality of decision making
suffers~\cite{gross1964managing}. 
%
In the context of social media, the high availability of diverse information may
prevent users to find relevant content.

%%

Finding relevant content in social media is not easily solved by conventional
search engines.
%
Publications on social media, or {\em posts}, can be of {\em variable quality}.
%
Posts are composed of multimedia pieces of content, but often they are brief or
short.
%
For instance, a post can be a very short text, an hyperlink, a single image or
video with little context.
%
They can be also irrelevant to the user's interest, for example, spam posts,
which contain relevant keywords but in a misleading way, in order to lure users
to a irrelevant website. 
%
Posts can also be out-dated, delivering incorrect or obsolete information. 
%
Many posts can be duplicate ones, published by automated agents, or by users
using ``share buttons'' in websites publishing posts with a template text; 
%
they can also be near-duplicates, with minor text differences, or sharing the
same resource from different URLs.
%
Another important characteristic of posts is that they are written in natural
language, so they can be incorrectly capitalized, misspelled, or with ambiguous
meaning.
%
Users make use of colloquial language and different forms of expression when
publishing content, e.g., abbreviations, hashtags (tags to describe content),
emojis (ideograms), etc.
%
Finally, messages can also be misleading, sharing false information.
%
All of these particularities of social media make it difficult to apply standard
techniques for users to find relevant content.

%%

At the same time, there is a massive scale of production of content. 
%
Twitter reports that there are about 320 million active monthly users and 500
million daily {\em tweets} or messages published in their
platform~\cite{twitter2014}. 
%
(These numbers were first stated in 2014 and have not been updated since then.)
%
Facebook has nearly 2.32 billion active monthly users~\cite{fbnewsroom}. 
%
Sina Weibo has 462 million active monthly users, while 93\% of them are on
mobile devices~\cite{chinawatch}.
%
The amount of content being published requires novel ways to deal with social
media data, in particular, news events.


%%

We regard news events as a higher level abstraction than single
posts.
%
In related work, an event\footnote{In this work, we will use the term {\em news
event} and {\em event} indistinctly.} is deemed as {\em something that happens
in a certain place and time}~\cite{yang1999learning}, while other definitions
consider an event as a collection of documents related to a certain
occurrence~\cite{Becker:2010:LSM:1718487.1718524}.
%
Throughout this dissertation, we will consider an event as a collection of
social media posts describing or commenting on a real-world occurrence.
%
In this sense, an event is a more complex piece of information compared to
single posts.
%
The notion of event is convenient as it is the base unit for new tasks, such as
event detection, tracking, or summarization.
%
Also, an event is comprised of posts of heterogeneous quality, from different
locations, and at different times.
%
This yields to new problems and challenges when studying social media.

%%

% PROPOSAL

\section*{Proposal}

In this dissertation, we tackle the problem of {\em extracting useful knowledge}
from events on social media. 
%
To be able to infer and extract useful information from events, we propose the
development of event representations that leverage specific features according
to the desired goal when analyzing social media data.
%
For this, we propose different models or representations of events based on
three perspectives:
%

\begin{enumerate}
    \item {\bf User activity.} 
    %
    When users react to an event, they may manifest this reaction on social
    media, producing or sharing content relevant to the event. 
    %
    The characteristics of these manifestations are dependent on the proper
    features of the occurrence, and not all are equal. 
    %
    We look at how the activity of users offers insight on the proper
    features of an event and incorporates this behavior in a compact
    representation.

    \item {\bf Spatio-temporal context.} 
    %
    Events develop in different locations. 
    %
    On the other hand, users from different locations may react differently to
    the same event.
    % 
    We study the development of events based on user activity conditioned by the
    location users are from, proposing a representation of events and locations
    based on social media posts.

    \item {\bf Common features in content.} 
    %
    Users may publish similar pieces of content in social media in reaction to
    events.
    %
    However, each post can contribute to a different aspect of the event, while
    sharing some common features.
    %
    We leverage these commonalities in content to produce a compact model that
    preserve topical information in events.
    
\end{enumerate}


% OBJETIVES
\section*{Objectives}

Our main objective is to define event representations through different data
aggregations to perform quantitative analysis of news events on social media.
%
Currently, it is very difficult to manage and analyze the high volume of
information being published when a event happens in the world. 
%
In particular, we study events through three perspectives: user reaction
and activity, spatio-temporal context of events, and content aggregation.
%
Understanding user reaction involves discriminating which events are more
important or produce more impact in a community. 
%
Analyzing spatio-temporal context refers to understanding how communities from
different locations are affected by different events, as seen on social media,
and identifying similar communities and events based on this context. 
%
Understanding content refers to the identification of the core aspects of an
event, without having to go through all the --potentially various-- posts.
%
In particular, our goal is to propose different models for representing events. 
%
These models should be flexible enough to apply diverse methodologies to
discover useful knowledge from information published on social media about
real-world events, from the perspectives described above. 


%%

% WHY TWITTER

We chose Twitter as our data source for this work. 
%
Twitter provides a simple way to obtain data and via its API (Application
Programming Interface~\cite{twitterapi}), from which we can obtain tweets
automatically and programatically. 
%
Furthermore, it is not as restrictive as other sources, such as Facebook, which
encourages users to maintain a private profile, hence making the data collection
much more difficult or impossible to perform.
%
Also, services like Facebook encourage users to share potentially personal
content, and not just event-related.
%
Twitter is primarily dedicated to encourage users to share event-related posts,
and it is mainly used as a news source by its users (about 70\% of its users use
Twitter to get news). 
%
For instance, its website asks ``What's happening?'' to users when publishing
content, as opposed to Facebook's ``What's in your mind?''
%
Therefore we think that Twitter is a suitable platform for this type of work.

% UTILITY

The study of news events on social media has several applications in the
proposed setting. 
%
How the community reacts to different events would allow us to identify the
characteristics of these events by a measure of reaction or other features given
by the social network or the content of the events. 
%
By these characteristics, it would be possible to identify or even predict which
events are going to cause a significant reaction from the community, hence
improving journalistic coverage or better responses from authorities facing an
emergency.
%
Additionally, by studying not only the response, but the context of different
communities and how they respond to certain events, may give us insights about
the communities themselves, for example, by revealing unexpected relations
between different communities, or by measuring event similarity using the
context, instead of content-based features. 
%
On the other hand, the study of the content is useful to understand the
different points of view ahead of an event. 
%
Users accustomed to the same perspectives given by other users or sources may be
oblivious of other angles of the same news event. 
%
Being capable of identifying the different aspects of an event and then presenting
these aspects in a concise summary can deal with this problem. 
%
All in all, the proposed framework can be of utility to understand social
behavior, study and decrease the effects of the information overload as well
as to perform comparative historical research~\cite{wiki:comparative}.

%%

\section*{Thesis statement} 
%
This dissertation defines flexible models for events on the social
networking platform Twitter. 
%
Having three perspectives in mind: user reaction, spatio-temporal context, and
content, the defined models should be able to allow us new insights
about news events reflected on Twitter. 
%
In particular, each perspective approaches specific Data Mining and Information
Retrieval techniques: 
%
the study of context involves modeling and exploratory data analysis; 
%
reaction involves filtering and classification;
%
content involves topic detection and document modeling.

The thesis statement is as follows:

{\bf \em
Modeling news events from user-contributed content on Twitter, based on
their spatio-temporal context, the reaction the users had on them, or the
multimedia content which the events contain, is novel and effective for performing
high-level quantitative analyses of news events.
}

% Statement of purpose

%\todo[inline]{statement of purpose}


\section*{Challenges} 

We identify three main challenges for this work:

\begin{itemize}
    \item {\bf Retrieval of relevant posts.} 
    Social media offers a partial view of the world. 
    %    
    Also, mainstream topics obfuscate distinct points of view, which can
    obstruct retrieval of diverse content. 
    %
    Because users are frequently posting messages about their own lives, daily
    situations, or general topics, trends can only be visible when looking at large
    volumes of data. 
    %
    This makes identification of events and relevant content a very difficult task.
    %
    And due to the characteristics of Twitter (or any other social networking
    service), messages are usually very short and with grammatical errors. 
    %
    Users also spontaneously create new ways to refer to the same entities
    (e.g., via the use of hashtags, emojis, or abbreviations), making it
    difficult to identify more relevant content when detecting events.
    %
    Relevant multimedia content is also difficult to identify. 
    %
    Multimedia content is represented by images, videos, text, or mix of them,
    etc. 
    %
    This information can be exploited to improve the effectiveness of
    the proposed methodologies. 
    %
    The challenge is just how to identify such content in an efficient way, how
    to deal with duplicated or quasi-duplicated content, and how to evaluate the
    effectiveness of methodologies when presenting multimedia content. 

    \item {\bf Data bias.}
    As stated above, social media offers a partial view of the world. 
    %
    Furthermore, the employed methodologies to retrieve or identify events from
    social media may be biased depending on several factors. 
    %
    For instance, we collected our dataset using news outlets as sources, being
    the majority of the outlets coming from the USA or the UK. 
    %
    Our sources also use specific words and ways to express the information,
    which can also create a bias in the way we further retrieve more tweets.
    %
    This is a huge challenge in order to provide generalizable results from the
    proposed methodologies. 
    %
    In addition, it is challenging to ensure that our results are as diverse as the
    utilized data source. 

    \item {\bf Validating results.} 
    As data in social media is being published at all times, it is unfeasible to
    apply standard measures such as Recall when evaluating a methodology,
    because we do not have all the relevant content available.
    %
    On the other hand, there are no {\em gold standards} to which we can
    contrast our models. 
    %
    We need to come up with methodologies to validate our results in order to
    provide generalizable results.

    
\end{itemize}


\section*{Contributions} 

There are four main contributions in this dissertation:

\begin{enumerate}
\item A novel event representation based on user activity triggered by news
events on Twitter. 
%
This representation allows us to rank events into different levels of activity. 
%
We also show that the activity can be determined by other event features, and
that these features appear early on the development of events.
%
We show that it is possible to {\em predict} early on the level of activity of
an event using aggregated post features.

\item A spatio-temporal representation based on the location where an event
happens, and the locations the users commenting on the news are from.
%
With this type of representation, we can compare events and locations based on
different factors, and track the evolution of an event based on the locations
involved in it.

\item A lightweight representation of content based on shared URLs.
%
We aggregate event-related posts based on common relevant URLs, retweets and
replies, generating a compact representation of an event.
%
In our preliminary experiments, we observed that the representation is one order
of magnitude smaller than the original data.
%
At the same time, we observed that with our representation we can achieve
comparable clustering results, with a fraction of running time and memory
required.

\item An event collection methodology based on {\em seed news outlets}.
%
Given a set of news outlets, every hour we extract the most relevant keywords
from their headlines and use them to retrieve relevant tweets from regular
users.
%
We also made available a dataset of 193 million tweets of $25,000$ news events,
from 2013 to 2015.
\end{enumerate}

Even though the different points of view posed as themes for this project cover
mostly independent approaches of event mining, they have in common the goal of
exploring and studying how different data aggregations can be useful for
extracting useful knowledge from events. 
%
This dissertation can be viewed as an exploration on how different data
aggregation strategies applied to events on social media are useful for easily
extracting knowledge or to serve as building blocks for new models and
methodologies.




\section*{Publications} 

The following publications are associated with this dissertation:

{\bf Journal papers.}
\begin{itemize}
    \item J. Kalyanam, {\bf M. Quezada}, B. Poblete, and G. Lanckriet. {\em
     Prediction and Characterization of High-Activity Events in Social Media
     Triggered by Real-World News}. In PLOS ONE 11(12): e0166694. 2016.

    \item V. Peña-Araya, {\bf M. Quezada}, B. Poblete, and D. Parra. {\em
    Gaining historical and international relations insights from social media:
    spatio-temporal real-world news analysis using Twitter.} In EPJ Data Science
    6, no. 1 (2017): 25. 2017.
\end{itemize}


{\bf Conference and Workshop papers.}

\begin{itemize}
    \item {\bf M. Quezada}, V. Peña-Araya, and B. Poblete. {\em Location-Aware
    Model for News Events in Social Media.} In Proceedings of the 38th
    International ACM SIGIR Conference on Research and Development in
    Information Retrieval (SIGIR '15). ACM, New York, NY, USA. 2015.
    (Short paper.)

    \item V. Peña-Araya, {\bf M. Quezada}, and B. Poblete. {\em Galean:
    Visualization of Geolocated News Events from Social Media.} In Proceedings
    of the 38th International ACM SIGIR Conference on Research and Development
    in Information Retrieval (SIGIR '15). ACM, New York, NY, USA. 2015.
    (Demo paper.)

    \item {\bf M. Quezada} and B. Poblete. {\em A Lightweight Representation of
    News Events on Social Media.} To be published in the 42nd International ACM SIGIR
    Conference on Research and Development in Information Retrieval (SIGIR '19).
    (Short paper.)

    \item {\bf M. Quezada.} {\em Knowledge Discovery of News Events in Social
    Media.} To be published within the CEUR-WS workshop proceedings, in the 9th
    PhD Symposium on Future Directions in Information Access (FDIA '19). 

\end{itemize}

%%

