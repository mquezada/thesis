\chapter{Conclusions}

In this dissertation we have proposed different representations of news events
from social media data.
%
Along with each representation, we also implemented some applications, showing
the effectiveness and usefulness of the representations.

\section{Summary of Contributions}

\paragraph{Model of User Reaction.}
%
We proposed an event representation that allows us to model events based on the
level of activity of users around news events. 
%
This model is based on the learned distribution of interarrival rates of social
media posts.
%
Using a dataset of about five thousand news events obtained from Twitter, we
computed the most frequent interarrival times of tweets, and used them to create
a vector representation of events based on these rates.
%
This means that our model is independent of the scale of the event.
%
We characterized the high-activity and low-activity events separately, finding
statistically significant differences between event types.
%
We also showed that other event features, such as the ratio of retweets or the
amount of words in the messages, are predictors of the level of activity of an
event, and that this level is very identifiable at early stages of the event.
%
The idea behind this representation was to model the impact that events cause in
the community, unlike notions such as virality, which is applicable to memes or
units of information, or popularity, which accounts for large-scale events.

\paragraph{Model of Spatio-Temporal Context.}
%
In a similar fashion, we proposed an event representation aiming for another way
to measure the impact, in this case, based on the locations from where users
comment on the news, and the locations where the event happened.
%
This allows us to explore how different locations are affected by different
events, and how users from different locations are interested in those events.
%
To show the effectiveness of this representation, we made an exploratory
analysis of a 2-year datasets of news events from Twitter, encompassing about
25,000 events, showing connections between locations and insights about
international relations using tweets.

\paragraph{Model of Aggregated Content.}
%
Lastly, we proposed a representation of content of news events, leveraging the
redundancy of information and the evidence that users share URLs when exposed to
a news event.
%
Our methodology generates a compact representation of events, allowing us to
perform standard tasks using less data, with similar results.
%
We showed how our representation could be used to find sub-topics in events,
comparing the sub-topics with those obtained by using all the data.
%
In our preliminary experiments, we showed that our model yields similar
clustering metrics compared to the complete data, by using one order of
magnitude less of vectors.


With respect to the thesis statement, we can say that the use of different types
of context information, such as the user reaction, the spatio-temporal setting
of the events, or the use of implicit information, such as the occurrence of
shared URLs, are {\em novel} and {\em effective} to perform analysis of news events. 
%
For instance, we observed that it is possible to determine if a news event is
going to produce high levels of activity on the community, meaning that there
are implicit signals given by the community as a whole, and not necessarily in
individual posts.
%
On the other hand, the modeling of the spatio-temporal context allows us to
infer new relationships between geopolitical entities (e.g., countries) using
social media data.
%
At long term, it can serve as a repository of historical data for further
analysis of political or societal trends\footnote{Consider, for example, the
applications to Comparative Historical Research.
\url{https://en.wikipedia.org/wiki/Comparative_historical_research} (Accessed:
August 24, 2019).}.
%
Finally, we showed a preliminary study of the potential of aggregating
information based on shared content, in this case about shared URLs.
%
The ability of posts to preserve topical information when aggregated by the same
shared URLs can be useful to, for instance, generate automatic summaries of
events, contributing to the posibility of storing this information for posterior
analyses.




\section{Future Directions}

Future work involves dealing with some of the limitations of our work, such as
the quality of our data collection methodology, the capabilities of our models
to be transferable to other social media platforms beside Twitter, and the steps
we need to perform in order to deal with biased data.

\subsection*{Data Collection}
% DATA
One of the future directions is related to the data extraction methodology
described in Chapter~\ref{chapter:data}. 
%
The news event extraction methodology relies on the headlines published by news
media accounts. 
%
This technique provides good precision in terms of reporting events that did in
fact exist in the real-world, but might omit informative events that did not
receive media coverage (unknown recall).
%
Therefore, the current data extraction approach can fail to retrieve events such
as citizen movements and other important events that were informed only via
social networks.  
%
In addition, in the current data extraction setup the initial seeds for the
event collection came from a reduced list of news media accounts, with limited
country coverage and languages.
%
Although the news event dataset likely represents a great majority of the news
events and related tweets posted on Twitter, the collection will miss the long
tail of events that had impact in other less represented countries worldwide. 
%
We note that there are several ways in which this bias can be mitigated in the
future, all of them related to replacing external modules in the data input
phase of the framework.
%
Another direction regarding this issue corresponds to merging events that
discuss the same news topic in different languages. 
%
Recent approaches in cross-language microblogging retrieval
\cite{Godavarthy2016} can be integrated for news event retrieval within our
framework.
%
Also, we will consider the posibility of incorporating higher-level temporal
properties of events, such as if the event is long-term, punctual or recurring,
as defined in the work of Tan et al.~\cite{st-model_2009}.


\subsection*{Generalization Potential in other Platforms}
% TRANSFERABILITY
In addition, we note that although our proposed event representations can be
considered generalizable to other social media platforms, we have not validated
it on other sources of information besides Twitter. 
%
It is not certain that for other social media platforms we will have enough
information, regarding user location and data availability, in order to produce
accurate event representations.
%
For example, it is possible to obtain similar results regarding the detection of
high-activity events, or the same topical results when aggregating content by
the same shared URLs?


\subsection*{Veracity and Ethics}
%
Having more data or the posibility to explore different platforms involve other
problems, such as the veracity of the information.
%
In this work we did not tackle this issue, assuming that all the events in our
dataset are trustworthy.
%
This problem raised an entire line of research in the last years.
%
Similar to this problem is the bias in the social platforms and the gathered
data.
%
We already performed some data normalization (e.g., in
Chapter~\ref{chapter:geopolitical}, we normalized the amount of messages in
order to avoid certain countries to be overrepresented), however, there are
implicit biases or biases that are dependant on demographic information, such as
age or gender~\cite{Graells-Garrido:2019:RAD:3292522.3326057}.
%
No platform is representative of the population, hence we need to incorporate a
sound methodology for normalizing data and be aware of the consequences of
polarization, bias, and misinformation present in social media data.