\chapter{Conclusions}

In this dissertation we have proposed different representations of news events
from social media data.
%
Along with each representation, we also implemented some applications, showing
the effectiveness and usefulness of the representations.
%
The summary of contributions is the following:

\paragraph{Model of User Reaction.}
%
We proposed an event representation that allows us to model events based on the
level of activity of users around news events. 
%
This model is based on the learned distribution of interarrival rates of social
media posts.
%
Using a dataset of about five thousand news events obtained from Twitter, we
computed the most frequent interarrival times of tweets, and used them to create
a vector representation of events based on these rates.
%
This means that our model is independent of the scale of the event.
%
We characterized the high-activity and low-activity events separately, finding
statistically significant differences between event types.
%
We also showed that other event features, such as the ratio of retweets or the
amount of words in the messages, are predictors of the level of activity of an
event, and that this level is very identifiable at early stages of the event.
%
The idea behind this representation was to model the impact that events cause in
the community, unlike notions such as virality, which is applicable to memes or
units of information, or popularity, which accounts for large-scale events.

\paragraph{Model of Geopolitical Entities.}
%
In a similar fashion, we proposed an event representation aiming for another way
to measure the impact, in this case, based on the locations from where users
comment on the news, and the locations where the event happened.
%
This allows us to explore how different locations are affected by different
events, and how users from different locations are interested in those events.
%
To show the effectiveness of this representation, we made an exploratory
analysis of a 2-year datasets of news events from Twitter, encompassing about
25,000 events, showing connections between locations and insights about
international relations using tweets.

\paragraph{Model for Shared Content.}
%
Lastly, we proposed a representation of content of news events, leveraging the
redundancy of information and the evidence that users share URLs when exposed to
a news event.
%
Our methodology generates a compact representation of events, allowing us to
perform standard tasks using less data, with similar results.
%
We showed how our representation could be used to find sub-topics in events,
comparing the sub-topics with those obtained by using all the data.
%
In our preliminary experiments, we showed that our model yields similar
clustering metrics compared to the complete data, by using one order of
magnitude less of vectors.

\todo{yes}


\section{Known Limitations}

\todo{yes}

There are several limitations that we consider important to address. 
%
In particular, these limitations are not related to our proposed event
representation, framework and implementation of our applications, but rather to
the data extraction methodology which depends on external functionalities.

The news event extraction methodology relies on the headlines published by news
media accounts. 
%
This technique provides good precision in terms of reporting events that did in
fact exist in the real-world, but might omit informative events that did not
receive media coverage.
%
Therefore, the current data extraction approach can fail to retrieve events such
as citizen movements and other important events that were informed only via
social networks.  
%
In addition, in the current data extraction setup the initial seeds for the
event collection came from a reduced list of news media accounts, with limited
country coverage and languages.
%
Although the news event dataset likely represents a great majority of the news
events and related tweets posted on Twitter, the collection will miss the long
tail of events that had impact in other less represented countries worldwide. 
%
We note that there are several ways in which this bias can be mitigated in the
future, all of them related to replacing external modules in the data input
phase of the framework.

In addition, we note that although our proposed event representation can be
considered generalizable to other social media platforms, we have not validated
it on other sources of information besides Twitter. 
%
It is not certain that for other social media platforms we will have enough
information, regarding user location and data availability, in order to produce
accurate event representations.

Overall, basic future improvements of our work should consider:
%
\begin{itemize}
\item Implementing automatic event detection techniques for Twitter based on the
data stream and network properties, as well as more comprehensive microblog
event extraction approaches.

\item Improving the geolocation tool accuracy. Despite CLAVIN's maturity as a
geolocation tool, it does not recognize location names in languages other than
English (even though the documentation of the tool indicates that it does
recognize alternative location names~\cite{clavin}).

\item Adding finer granularity to the geographical context extractor of our
system, in order to include more precise administrative divisions such as cities
and states.

\item Merging events that discuss the same news topic in different languages.
Recent approaches in cross-language microblogging retrieval
\cite{Godavarthy2016} can be integrated for news event retrieval within our
framework.
\end{itemize}

All of these improvements are however beyond the current scope of our work,
which focuses on providing proof of the usefulness of the proposed event
representation. 
%
Nevertheless, we are working on improving all of these features in future
versions of our applications.  
%
For example, we have already started the task of providing more fine-grained
locations for Chile and comprehensive sets of local news sources as in the work
of Maldonado et al.~\cite{maldonado2015spatio}.

\section{Future Directions}

\todo{yes}


In the future we will extend our representation to incorporate higher-level
temporal properties of events, such as if the event is long-term, punctual or
recurring, as defined in the work of Tan et al.~\cite{st-model_2009}.
%

Another interesting line of research for future work is that of studying how
pairwise similarities between countries evolve over time. 
%
This could be useful to automatically discover when new relationships arise and
when deviations from normal patterns occur.

Finally, we are researching techniques to improve data extraction for news
events, based on techniques such as those reported by Hasan et
al.~\cite{doi:10.1177/0165551517698564}, to achieve a better representation of
different locations, and to improve geolocation. 
%
We are also working towards creating automatic summaries of events in social
media, on how to track and visualize event evolution over time. 
%
In particular the relations among protagonist countries that happen as
consequence of these events, as well as incorporating approaches of
cross-language information retrieval\cite{grefenstette2012cross} for news event
recommendations in microblogs.