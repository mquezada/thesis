
\section{Introduction}\label{sec:geo-introduction}

The immense growth of the social Web, which has made a large amount of user data
easily and publicly available, has opened a whole new spectrum for research in
social behavioral sciences.  
%
However, as the volume of social media content increases at a very fast rate, it
becomes extremely difficult to systematically obtain high-level information from
this data.  
%
As a consequence, tasks related to the analysis of historical news events based
on social media data have not been explored, which limits any type of
comparative historical research, causality analysis, and discovery of knowledge
from patterns extracted from aggregated social media event information.

In this chapter we target this issue by proposing a compact high-level
representation of news events using social media information. 
%
This representation explicitly includes temporal information about the event and
information about locations, in particular of geopolitical entities. 
%
We call this a {\em spatio-temporal context-aware event representation}. 
%
Our hypothesis is that by including social, temporal, and spatial information in
the event representation, we are enabling the analysis of historical world news
from a social and geopolitical perspective.  
%
This facilitates, new information retrieval tasks related to historical event
information extraction and international relations analysis. 
%
We support our claims by presenting a quantitative analysis of a 2-year Twitter
dataset of news events reported by U.S. and U.K.  media, which we explore using
data mining techniques on our event representations. 
%
We also show how using a visual tool, named Galean, we can explore and retrieve
historical news events within their geopolitical and temporal context.



As social media becomes massively popular, it is used as a reliable and
efficient news source.  
%
Many users turn to social media platforms to obtain information, especially
breaking news~\cite{Rogers:2013:DTT:2464464.2464511}. 
%
Even traditional mass media organizations such as newspapers and TV news
channels now use social media platforms to inform their audience more quickly.
%
Social media users are not only consumers of this information, but also
producers and broadcasters. 
%
Millions of people from all over the world have assumed the task of reporting
and commenting on newsworthy events.  
%
In particular, the social platform Twitter has become a preferred source for
users to find up-to-date information.  
%
When breaking news occurs, Twitter users quickly react by generating content and
producing interactions.  
%
The particular nature of Twitter messages, as well as the fact that most of its
users use the platform from mobile devices, facilitates extremely fast
information propagation.

%%% NEED - why something needed to be done at all
%
Twitter provides excellent conditions for social behavior analysis, as well as
comparative historical research, among many other social and scientific
disciplines.  
%
In particular, the field of {\em comparative historical
research}\footnote{\url{https://en.wikipedia.org/wiki/Comparative_historical_research}
(Accessed: June 19, 2019.) See also
\url{https://en.wikipedia.org/wiki/Comparative_research} (Accessed: June 19,
2019)} examines historical events in comparison to other historical events to
gain general knowledge that goes beyond a particular event. 
%
So far, historical research had been restricted to traditional archival data and
historians' written account of past events. 
%
Nevertheless, it is undeniable that the data poured into social media about
world events is of great value to society. 
%
The evidence is in the increasing body of scientific work surrounding retrospective
microblog data.  
%
Just to mention a few examples: Castillo et al.~\cite{castillo2011information}
extracted information to predict the credibility of rumors in social media,
Sakaki et al.~\cite{Sakaki2010} used Twitter for real-time earthquake detection,
Pak and Paroubek~\cite{Pak:Twitter:2010} studied Twitter messages as a corpus
for sentiment analysis, and Saravanou et al.~\cite{Saravanou:Twitter:2015} used
tweet coordinates to find locations that have been affected by floods.

%%

Despite the usefulness of historical information extracted from social media,
there is not much research addressing the topic of retrospective analysis of
this data.
%
Social media in general, and Twitter in particular, produce streaming data that
is volatile, which most likely explains why existing research concentrates only
on a particular event.

%%% Task - what the we undertook to address the need (first person, past
%%% tense)
We addressed this issue by introducing novel data mining tasks, based on a
compact representation of real-world news events. 
%
This representation was designed to summarize information about real-world
events from social media data, which is enhanced with the event's geographical
and temporal context.

%%

Our event representation incorporates two types of spatial data about an event:
%
1) locations directly {\em involved} in the real-world event occurrence (i.e.,
the main places that are mentioned in messages about the event), which we refer
to as {\em protagonist locations}, and 
%
2) locations from where social network users {\em comment} on the event (i.e,
the places where users that comment are located), which we refer to as {\em
interested locations}.  
%
For example, when an earthquake took place in Nepal in April, 2015, most of the
messages mentioned Nepal, which indicated that this was the location where the
event had taken place. 
%
Therefore, if we consider locations at a country level, Nepal can be regarded as
the {\em protagonist location} of that event. 
%
However, the users that posted the messages about Nepal were distributed all
over the world, indicating that this event had global impact. 
%
Furthermore, some countries had more users interested in the event than other
countries, such as, neighboring countries and countries with citizens among the
victims.  
%
These would be considered as the {\em interested locations} of that event.

%%

Our work is based on the hypothesis that by adding spatio-temporal context to
news events, such as protagonist and interested locations, and the time at which
it occurred, we can discover new information based solely on social media data.
%
In particular, the application of our event representation allows us to find
relationships among events and among locations, such as:

\begin{enumerate}
\item{\bf event similarity:}
  \begin{itemize}
  \item {\bf based on their protagonist locations}, i.e., retrieve all the
    events that occurred in certain location, or that directly involved similar
    groups of locations;
  \item {\bf based on the locations that are interested in the event}, i.e.,
    retrieve all of the events that produced the similar interest in other
    locations.
  \end{itemize}
\item{\bf location similarity:}
  \begin{itemize}
  \item {\bf based on events in which a location is protagonist}, i.e., retrieve
the locations that are protagonists in the same events;
  \item {\bf based on their interest in events}, i.e.,  retrieve sets of
locations that showed similar levels of interest in the same events.
  \end{itemize}
\item{\bf any combination of the above.}
\end{enumerate}

These similarity relationships along with temporal context can facilitate the
implementation of novel information retrieval tasks. 
%
These tasks include: event search, event understanding, geopolitical analysis,
international relations analysis (when considering locations at a country
level), historical comparative analysis, among others.

\paragraph{Contributions.} Our contributions are the following:

\begin{enumerate}
\item We present a novel high-level representation of news events based on
information extracted from social media. This representation emphasizes the
geographical and temporal context of news;
\item We present an exploratory analysis of a 2-year data collection in which we
use our proposed event representation to identify connections and similarity
patterns among countries; and
\item We show how a visual tool, named Galean, designed for exploration of
historical news event collections based on our proposed approach, allows users
to view event evolution over long periods of time, the relationship between the
geopolitical entities that participate in them, and emergent patterns.
\end{enumerate}

An application of the model and visual tool proposed in this chapter and
corresponding publications can be found in the work of Maldonado et
al.~\cite{maldonado2015spatio}.
%
They show a version of Galean specifically designed for exploring Chilean news.
%
An online version of this application can be found in \url{http://galean.cl/}
(Accessed: June 19, 2019).