\section{Known Limitations}\label{sec:limitations}
\todo[inline]{poner esto en las conclusiones o en el chap3}
There are several limitations that we consider important to address. 
%
In particular, these limitations are not related to our proposed event
representation, framework and implementation of our applications, but rather to
the data extraction methodology which depends on external functionalities.

The news event extraction methodology relies on the headlines published by news
media accounts. 
%
This technique provides good precision in terms of reporting events that did in
fact exist in the real-world, but might omit informative events that did not
receive media coverage.
%
Therefore, the current data extraction approach can fail to retrieve events such
as citizen movements and other important events that were informed only via
social networks.  
%
In addition, in the current data extraction setup the initial seeds for the
event collection came from a reduced list of news media accounts, with limited
country coverage and languages.
%
Although the news event dataset likely represents a great majority of the news
events and related tweets posted on Twitter, the collection will miss the long
tail of events that had impact in other less represented countries worldwide. 
%
We note that there are several ways in which this bias can be mitigated in the
future, all of them related to replacing external modules in the data input
phase of the framework.

In addition, we note that although our proposed event representation can be
considered generalizable to other social media platforms, we have not validated
it on other sources of information besides Twitter. 
%
It is not certain that for other social media platforms we will have enough
information, regarding user location and data availability, in order to produce
accurate event representations.

Overall, basic future improvements of our work should consider:
%
\begin{itemize}
\item Implementing automatic event detection techniques for Twitter based on the
data stream and network properties, as well as more comprehensive microblog
event extraction approaches.

\item Improving the geolocation tool accuracy. Despite CLAVIN's maturity as a
geolocation tool, it does not recognize location names in languages other than
English (even though the documentation of the tool indicates that it does
recognize alternative location names~\cite{clavin}).

\item Adding finer granularity to the geographical context extractor of our
system, in order to include more precise administrative divisions such as cities
and states.

\item Merging events that discuss the same news topic in different languages.
Recent approaches in cross-language microblogging retrieval
\cite{Godavarthy2016} can be integrated for news event retrieval within our
framework.
\end{itemize}

All of these improvements are however beyond the current scope of our work,
which focuses on providing proof of the usefulness of the proposed event
representation. 
%
Nevertheless, we are working on improving all of these features in future
versions of our applications.  
%
For example, we have already started the task of providing more fine-grained
locations for Chile and comprehensive sets of local news sources as in the work
of Maldonado et al.~\cite{maldonado2015spatio}.



\section{Conclusions and Future Work} \label{sec:conclusions}


We have presented a spatio-temporal context-aware representation for news events
in social media. 
%
Using this representation we have introduced a visual analytics tool named
Galean that allows for retrospective analysis of real-world events through the
aggregation of information posted by social media users. 
%
The main goal of our tool and our event representation is to allow exploration
and quantitative analysis of events from a geographical and temporal
perspective. 
%
In particular we introduce two types of geographical contexts for events: 1)
protagonist locations, and 2) interested locations. 
%
The first corresponds to locations, in this case geopolitical divisions, that
were involved in the event itself, and the second corresponds to locations where
the event's information had the most impact.


In addition, we introduce a quantitative data mining study over a 2-year Twitter
dataset, in which we explore the properties of news events in social media and
the international relations that are induced by those events.  
%
Our findings indicate that indeed there is new information, which can be
extracted at large scale, about how countries relate and how information is
perceived in different places. 
%
Most interestingly, these relationships reflect historical relations that are
found in the real world, indicating that there is value in social media data for
historical research.
%
Overall, our representation allows us to perform new IR tasks, related to
exploring international relations and historical event retrieval.

\todo[inline]{future work aca o al final?}

In the future we will extend our representation to incorporate higher-level
temporal properties of events, such as if the event is long-term, punctual or
recurring, as defined in the work of Tan et al.~\cite{st-model_2009}.
%

Another interesting line of research for future work is that of studying how
pairwise similarities between countries evolve over time. 
%
This could be useful to automatically discover when new relationships arise and
when deviations from normal patterns occur.

Finally, we are researching techniques to improve data extraction for news
events, based on techniques such as those reported by Hasan et
al.~\cite{doi:10.1177/0165551517698564}, to achieve a better representation of
different locations, and to improve geolocation. 
%
We are also working towards creating automatic summaries of events in social
media, on how to track and visualize event evolution over time. 
%
In particular the relations among protagonist countries that happen as
consequence of these events, as well as incorporating approaches of
cross-language information retrieval\cite{grefenstette2012cross} for news event
recommendations in microblogs.