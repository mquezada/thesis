
\section{Conclusion} \label{sec:conclusions}


We have presented a spatio-temporal context-aware representation for news events
in social media. 
%
Using this representation we have introduced a visual analytics tool named
Galean that allows for retrospective analysis of real-world events through the
aggregation of information posted by social media users. 
%
The main goal of our tool and our event representation is to allow exploration
and quantitative analysis of events from a geographical and temporal
perspective. 
%
In particular we introduce two types of geographical contexts for events: 1)
protagonist locations, and 2) interested locations. 
%
The first corresponds to locations, in this case geopolitical divisions, that
were involved in the event itself, and the second corresponds to locations where
the event's information had the most impact.


In addition, we introduce a quantitative data mining study over a 2-year Twitter
dataset, in which we explore the properties of news events in social media and
the international relations that are induced by those events.  
%
Our findings indicate that indeed there is new information, which can be
extracted at large scale, about how countries relate and how information is
perceived in different places. 
%
Most interestingly, these relationships reflect historical relations that are
found in the real world, indicating that there is value in social media data for
historical research.
%
Overall, our representation allows us to perform new IR tasks, related to
exploring international relations and historical event retrieval.
