\section{Limitations}\label{sec:geo:limitations}

There are some limitations in our proposed applications which we consider
important to address. In particular, these limitations are not related to our
proposed event representation, framework and implementation of our applications,
but rather to the data extraction and toponym extraction and annotation
methodology which depends on external tools.

Overall, basic future improvements of our work should consider the following
aspects:

\begin{itemize}
\item The incorporation of automatic event detection techniques for Twitter
based on properties of the data stream and network themselves, as well as more
comprehensive microblog event extraction approach. 
\item Improvements to the geolocation tool accuracy, this is important because
despite that CLAVIN is a powerful tool it does not recognize location names in
languages other than English (even though the documentation of the tool
indicates that it does recognize alternative location names~\cite{clavin}.
\item Adding finer granularity to the geographical context extractor of our
system in order to include administrative division such as cities and states.
\item Multi-language integration for merging events in different languages that
discuss a same news topic. Recent approaches in cross-language microblogging
retrieval \cite{Godavarthy2016} can be integrated for news event retrieval
within our framework.
\end{itemize}

However, all of these improvements are beyond the current scope of this work,
which has been focused on providing proof of the usefulness of our proposed
event representation. Nevertheless, we have already started the task of
providing more fine grained locations for Chile and comprehensive sets of local
news sources as well in~\cite{maldonado_2015}.

Regarding our visualization tool, we note that even though it is an event
retrieval tool, it is currently not focused on event ranking nor tweet ranking.
These functionalities are not provided as the focus has been placed on event
exploration within spatio-temporal filters. In the future, event and tweet
ranking functionalities could be incorporated as optional, using
state-of-the-art work on those areas. At the moment, the full display of events
and chronological ordering of tweets, appears to work well for event
exploration.
