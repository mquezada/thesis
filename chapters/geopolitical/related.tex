\subsection{Background}\label{sec:related}

Certain studies focused specifically on the task of detecting events and tagging
their relevant geo-locations. 
%
In particular, some works targeted the detection of localized
events~\cite{Watanabe:Jasmine:2011,Abdelhaq:EvenTweet:2013,Walther:2013fb,Lee:A:2011,Krumm:2015},
others the detection of global events~\cite{sankaranarayanan:twitterstand:2009},
and the detection of critical
events~\cite{Sakaki:Tweet:2013,DeLongueville:2009}.  
%
Dong et al.~\cite{Dong2015}, specifically, considered that events had different
temporal and spatial scales and proposed a multi-scale event detection approach
for social media. 
%
This approach focuses on detecting and reporting events with geo-localization.
%
Our current approach differs from existing work, in that we create an aggregated
representation of the information about real-world events, producing a
high-level representation that includes the event's geographical context, which
is extracted from social media. 
%
In addition, we enrich the information about an event by using the locations of
the users that post information about it.

Wang et al.~\cite{Wang:LeadLine:2012} visualized topics based on the extraction
of geographical entities from tweet text. 
%
They did not use this information to establish the location of an event, but
rather for event exploration. 
%
SensePlace2~\cite{MacEachren:SensePlace2:2011} is a Visual Analytics tool that
allows users to explore a set of tweets and models them by showing two
geographical types of information: the locations from where users discussed the
topic and the locations being mentioned in tweets. 
%
However, unlike our work, this information was only used at single tweet level,
and not at event level.

In the domain of cyber-physical systems, {\em events} are viewed as conditions
of interest~\cite{st-model_2009} within a cyber-physical system, or as the
co-occurrence of two people in the same physical place~\cite{STEvent_2010}.
%
In general, events are modeled according to the state of the objects in the
system, considering attributes, time and location. 
%
The work presented by Tan et al.~\cite{st-model_2009} bears certain similarities
with our own, in the sense that they considered an event to encompasses multiple
information about a condition of interest in the system (in our case in the
online social network), including time and physical locations. 
%
In addition, the authors defined different kinds of temporal and geographical
scopes for their events, which are similar to our definition of {\em event
impact}. 
%
The main difference relies in that our approach aims to capture high-level
information of how a complex exogenous event, such as a news event, is perceived
by social network users in an aggregated way. 
%
Therefore, we focus on geopolitical divisions as units of aggregated spatial
information and on representing geopolitical interactions.

%%

Despite that the idea of adding spatio-temporal context to social media data is
not novel, to the best of our knowledge our work is the first that formally
introduces {\em protagonist} and {\em interested} locations in a high-level
event representation.  
%
The novelty of our approach relies on the extension of the notion of spatial
context, first by associating real-world news to one or more protagonist
locations, and second by associating real-world news to the locations where they
generated interest.  
%
In addition, our work does not focus on event detection, classification or
summarization, as most of the prior work on event analysis does.

\paragraph{Quantitative Historical Event Analysis.}
We provide a revision of the literature on {\em quantitative history} research
applied to event analysis and social media. Quantitative history is an approach
to historical research that makes use of quantitative and digital
tools. 
%
To the best of our knowledge, our work is among the first to make use of social
media data for quantitative historical research.

Prior work used digitized newspapers and books for extracting quantitative
knowledge~\cite{Michel176,leetaru2011culturomics,chadefaux2014early}. 
%
Michel et al.~\cite{Michel176} built a corpus of 5 million books and analyzed
them using word frequencies to investigate cultural trends, and called this type
of study ``Culturomics.''
%
Leetaru~\cite{leetaru2011culturomics} performed a large-scale study of 30 years
of digitized newspapers.
%
Chadefaux~\cite{chadefaux2014early} used a dataset from Google News Archive to
predict military conflicts.

A different line of research covers digitized writings and the Semantic Web.
%
Suchanek and Preda~\cite{Suchanek:2014:SC:2732977.2732994} proposed the study of
``Semantic Culturomics'', in which the analysis of newspapers should go beyond
the study of word frequencies in order to integrate knowledge bases (such as
DBPedia~\cite{dbpedia}) to answer complex user queries. 
%
Additional research has used knowledge bases along with human writings, such as
newspapers~\cite{Huet:2013:MHL:2509558.2509567,DS/CN175}. 
%
A survey on this topic is provided by Mero\~no-Pe\~nuela et
al.~\cite{merono2014semantic}.

Compared to prior work, our approach among the first to consider user-generated
information networks, such as online social networks, which are a growing data
source at much larger scale.  
%
We consider that social media can provide additional and novel information to
that found in news articles and books. 
%
User-generated content reflects social opinions and points of view related to
current world-events. 
%
This content is generated in real-time, it is not edited and does not depend on
the editorial lines of formal news outlets.  
%
We believe that these unique characteristics make social media a challenging and
valuable source of historical information.  
%
Our approach incorporates the content of social media platforms about real-world
news, as well as aggregated geographical information that conveys the importance
and scope of these events.