\chapter{Related Work}

In this chapter we present the state of the art referring the design and
development of news event models in social media to perform high-level
quantitative analysis of events.

%%

High-level quantitative analysis of human-generated content has already been
approached in the past. 
%
{\em Culturomics} is defined as {\em the application of high-throughput data
collection and analysis to study of human culture}~\cite{Michel176}. 
%
The term was first introduced in a {\em Science} article from 2010, where the
authors studied the contents of about 5 million digitized
books~\cite{Michel176}. 
%
Our focus is concerned to the development of information models in the context
of news events for studying trends and culture.
%
In the sense of information models, Suchanek and
Preda~\cite{Suchanek:2014:SC:2732977.2732994} proposed the term ``Semantic
Culturomics,'' or the use of knowledge bases and Semantic Web techniques to give
structure to human-generated data to obtain new insights.
%
On the other hand, Leetaru~\cite{leetaru2011culturomics} proposed the
application of culturomics to news events, applying standard Natural Language
Processing techniques to traditional media, finding insights about international
relations.
%
GDELT~\cite{leetaru2013gdelt} is one example of the application of culturomics
to the traditional news media: it is a collection of news events with
automatically added metadata, such as locations, sentiment, entities, etc.

%%

In the context of social media, there are scattered attempts to create an
unified system to collect and annotate events, mainly due to the noisy nature of
user-contributed messages.
%
The presence of irrelevant, redundant, or untrustworthy messages make it
difficult to have robust models in social media.
%
Most of the representations proposed in the literature are tailored to specific
applications in Information Retrieval, specifically, in Topic Detection and
Tracking~\cite{allan2012topic} (TDT) tasks.
%
On social media, studies on news events have been specific to diverse tasks,
such as event detection, tracking, prediction, sentiment analysis,
summarization, and virality. 

In this chapter we review some of these models and applications in social media.


\section{Event Models in Social Media}

Most of the research in social media event analysis has been directed towards
the creation of event models for specific tasks such as detection, tracking,
summarization and characterization of events in social media streams. 
%
Typically, event information is represented as bag of words, {\em tf-idf}
vectors~\cite{tfidf,Marcus:2011:TAV:1978942.1978975}, language
models~\cite{zellers2019neuralfakenews}, probability distributions over
text~\cite{o2010tweetmotif,Hong:2010:EST:1964858.1964870,zhao2011comparing,Mehrotra:2013:ILT:2484028.2484166},
or graph
representations~\cite{Setty:2018:ENE:3209978.3210136,Lee:2013:KSK:2487575.2487711,Lee:2014:CCS:2661829.2661859}.
%
Some other works exploit domain-specific features, such as the occurrence of
hashtags~\cite{Kamath:2013:SDO:2488388.2488447},
URLs~\cite{Alonso:2015:WCW:2740908.2745397},
locations~\cite{Abdelhaq:EvenTweet:2013}, or a mixture of them~\cite{castillo2011information}.
%
Below we mention different approaches instead of pure representations, as they
are designed mostly for the target application.
%


\paragraph{Event Detection.} 
%
The goal of event detection on social media is to identify coherent
time-constrained topics from the content that is continuously being published.
%
The majority of the methods for event detection on Twitter rely on frequency
analysis, semantic analysis, graph modeling, machine learning methods, or a
mixture of them.
%
For example, the work of Sakaki et al.~\cite{Sakaki2010} attempts to develop a
system for earthquake detection using Twitter.
%
The authors design a classifier to detect if a tweet is talking about an
earthquake or not, using several features related to time, location, and
information difussion between users.
%
Another example is the work of Poblete et al.~\cite{poblete2018robust}, which
detects changes in tweet volume to detect earthquakes.
%
TwitInfo~\cite{Marcus:2011:TAV:1978942.1978975} is a system to detect events on
Twitter, based on a set of features, such as content (bag of words and tf-idf
vectorization), sentiment values, and time (detecting peaks in tweet volume).
%
Semantic methods involve the study of the relationship between n-grams in the
text to detect topics or events, such as the work of Cai et
al.~\cite{cai2015popular}, which relies in topic models (such as Latent
Dirichlet Allocation, or LDA, and its variants) to detect events.
%
Graph modeling approaches attempt to create a graph representation of messages
and then find appropiate sub-graphs that resembles events, such as the work of
Lee et al.~\cite{Lee:2013:KSK:2487575.2487711,Lee:2014:CCS:2661829.2661859}.
%
Machine learning methods, such as clustering and classification, are also used
to detect events.
%
Becker et al.~\cite{Becker:2010:LSM:1718487.1718524} proposed an incremental
clustering approach that groups messages if they are similar above a threshold.
%
This threshold is set empirically, but the features used for similarity are fine
tuned for the application.
%
TwitterStand~\cite{Sankaranarayanan:2009:TNT:1653771.1653781} is a system
designed using similar principles as the previous work mentioned.
%
Certain studies focused specifically on the task of detecting events and tagging
their relevant geolocations.  
%
In particular, some works targeted the detection of localized
events~\cite{Watanabe:Jasmine:2011,Abdelhaq:EvenTweet:2013,Walther:2013fb,Lee:A:2011,Krumm:2015},
others the detection of global events~\cite{sankaranarayanan:twitterstand:2009},
and the detection of critical
events~\cite{Sakaki:Tweet:2013,DeLongueville:2009}.
%
Dong et al.~\cite{Dong2015}, specifically, considered that events had different
temporal and spatial scales and proposed a multi-scale event detection approach
for social media.
%
This approach focuses on detecting and reporting events with geolocalization.
%
Another approach is to use information extraction techniques to identify named
entities, location names, etc. from Twitter
data~\cite{Ritter:2012:ODE:2339530.2339704}.
%
Temporal features for events have been used in tasks such as the detection of
events based on the temporal dynamics of their mentions in social
media~\cite{guille2015event}, and also for event
categorization~\cite{Ritter:2012:ODE:2339530.2339704}. 
%
The task in mind in this case is to perform event linking, that is, to match the
same news from different sources.
%
A survey of event detection methods on Twitter can be found
in~\cite{hasan2018survey}.



\paragraph{Event Summarization.} 
%
Event summarization methods aim to reduce the amount of information in events,
by fulfilling certain criteria, such as relevance (the summary has to be relevant
to the event), diversity or coverage (the same aspects of the event must be
present in the summary), conciseness (to be human readable), and so on.
%
In this case, current methods attempt to select the most relevant pieces of
content in order to create a summary\footnote{This is known as {\em extractive
summarization}, as opposed to {\em abstractive summarization}, which creates its
own description of the event, not necessarily using content already in the
data.}, by identifying relevant aspects of the event and then selecting the most
important pieces of content.
%
Chakrabarti and Punera~\cite{chakrabarti2011event}, for example, used hidden
Markov models to represent sub-events, within a broader event that is described
using Twitter data. 
%
This model identified the corresponding time-windows when something relevant
happened, and then using standard cosine similarity over the tf-idf vectors to
extract the most relevant tweets.
%
Similarly, Alsaedi et al.~\cite{alsaedi2016temporal} proposed a way to process
large amounts of data by splitting the dataset into time-framed segments, and
then summarizing each frame using a tf-idf based approach.
%
Another approach was presented in a previous
work~\cite{quezada2013understanding}, which focused on automatic summarization
of multimedia content by using social media posts as surrogate text for
multimedia documents.
%
A similar approach was used by Alonso et
al.~\cite{Alonso:2015:WCW:2740908.2745397}, which was based on the \emph{social
signature} of documents (that is, the set of keywords of social media messages
that point to a document), to augment the document information.
%
MGraph~\cite{schinas2016mgraph} was designed to detect topics on images and text
simultaneously from Twitter event data, and extracting the most important posts
and images to create a summary.
%
The authors represented posts and images as a similarity graph and then ranked
the most relevant pieces of content.
%
Likewise, Xu and Lu~\cite{Xu:2015:SBP:2678025.2701385} represented Tumblr posts
as a graph and then ranked the posts in order to extract a summary.
%
Bian et al.~\cite{bian2014multimedia} represented events using graphical models
with textual and visual features of text and images from Sina Weibo,
respectively.
%
Similarly, the output of their model is the set of relevant images and texts,
which fulfill their criteria.

\paragraph{Exploratory Analyses and Systems.}
%
In the work of Kamath et al.~\cite{Kamath:2013:SDO:2488388.2488447}, the authors
analyzed Twitter \emph{hashtags} in a large-scale study of the spatio-temporal
dynamics of {\em memes}, or ``units'' of information. 
%
In this work, a hashtag was represented as a tuple consisting of the coordinates
of the hashtag's location over time. 
%
They used a simple model to find interesting insights about the adoption and
spread of memes in social media. 
%
Memes are information which emerges from social networks and spreads in a viral
way. 
%
However, meme dissemination does not necessarily resemble how other types of
information will propagate, such as information about events that originate
outside of the network (i.e., exogenous events), or that originate from
different sources at the same time.
%
Saravanou et al.~\cite{Saravanou:Twitter:2015} identified areas of interest
after floods in the UK using tweets, by clustering messages based on their
geographical coordinates and then selecting the most relevant messages from the
most dense areas.
%
Alonso et al.~\cite{Alonso:2017:WHH:3091478.3091484} designed a system for
exploring popular trends. 
%
The authors represented collections of tweets as diffussion trees, by
aggregating tweets by shared URLs and then computing popularity metrics of each
tree, and finally presenting the most relevant URLs with corresponding tweets.
%
Wang et al.~\cite{Wang:LeadLine:2012} visualized topics based on the extraction
of geographical entities from tweet text. 
%
They did not use this information to establish the location of an event, but
rather for event exploration.
%
SensePlace2~\cite{MacEachren:SensePlace2:2011} is a Visual Analytics tool that
allows users to explore a set of tweets and models them by showing two
geographical types of information: the locations from where users discussed the
topic and the locations being mentioned in tweets. 
%
However, this information was only used at single tweet level.


\paragraph{Social Media and News Media.}
%
Related work has also approached the analysis of social media data to support or
improve news delivery. 
%
For example, \v{S}tajner et al.~\cite{Stajner:2013:ASS:2487575.2487659} proposed
an optimization problem to select the most interesting response to a news
article, where responses are tweets.
%
Diakopoulos et
al.~\cite{Diakopoulos:2012:FAS:2207676.2208409,diakopoulos2010diamonds} designed
a system to explore news events with journalists as the target users.
%
The authors propose different features as they deemed relevant for their use
case: relevance, uniqueness, sentiment, verity, etc. 
%
The study of the spread of rumors, ``fake news'' and the veracity of content is
a very active field to the date of this dissertation.
%
One of the first studies tackling this problem is the work of Castillo et
al.~\cite{castillo2011information}, where the authors investigate the
characteristics of rumors as they spread in Twitter, and developed a classifier
using several content, network, and user-based features.



\section{Quantitative Analysis of News Events}

In a different type of study, Leetaru~\cite{leetaru2011culturomics} performed a
large-scale analysis of 30 years of digitized news articles.  
%
The author computed sentiment scores and geolocation for each article.
%
The study indicated that some critical events in the past, such as social
revolutions, could have been forecasted by looking at sentiment scores over
time.
%
In addition, the author performed community detection on country graphs by
analyzing news in which two or more countries were involved.  
%
In this sense, our approach is similar, because we model countries in terms of
their co-occurrence in news. 
%
However, our work is focused on automatic information extraction from online
social streams and on the creation of a more general representation.  
%
We do not focus on the analysis of sentiment of edited content from formal news
media outlets, but on the interactions between locations, based on the
aggregated reactions and opinions of users of social platforms.
%

Castillo et al.~\cite{Castillo:2014} studied the traffic
that online news sources receive in response to news, finding two types of news
articles, based on their ``shelf-life'' in social media responses. 
%
Their findings are similar to ours when analyzing the characteristics of
high-activity events in Chapter~\ref{chapter:high-activity}.


In the domain of cyber-physical systems, {\em events} are viewed as conditions
of interest~\cite{st-model_2009} within a cyber-physical system, or as the
co-occurrence of two people in the same physical place~\cite{STEvent_2010}.
%
In general, events are modeled according to the state of the objects in the
system, considering attributes, time and location.
%
The work presented by Tan et al.~\cite{st-model_2009} bears certain similarities
with our own, in the sense that they considered an event to encompasses multiple
information about a condition of interest in the system (in our case in the
online social network), including time and physical locations. 
%
In addition, the authors defined different kinds of temporal and geographical
scopes for their events, which are similar to our definition of {\em event
impact}. 
%
The main difference relies in that our approach aims to capture high-level
information of how a complex exogenous event, such as a news event, is perceived
by social network users in an aggregated way. 
%
Therefore, we focus on geopolitical divisions as units of aggregated spatial
information and on representing geopolitical interactions.

The work of Setty and Hose~\cite{Setty:2018:ENE:3209978.3210136} aims to create
a common representation of events based on neural embeddings, on a procedure
similar to Node2Vec~\cite{Grover:2016:NSF:2939672.2939754}.
%
A general representation of events is useful to perform high-level quantitative
analysis, as it is flexible enough for arbitrary events.
%
However, it suffers from interpretability, as it is very difficult to determine
the underlying factors that produced certain representation.
%
Sellam and Alonso~\cite{10.1007/978-3-319-19890-3_17} designed a system for
information extraction from news events on Twitter that extracts quantitative
data from events, such as dates, numbers, entities, etc.
%
