\chapter{State of the Art}


\section{Event Models in Social Media}

Most of the research in social media event analysis has been directed towards
the creation of event models for specific tasks such as detection, tracking, summarization
and characterization of events in social media streams. However, not much work has
been focused on high-level event modeling with context information, such as
spatio-temporal information.


In the work of Kamath et al.~\cite{Kamath:Spatio-temporal:2013}, Twitter
\emph{hashtags} (i.e., user-generated
string prefixed by \# that users add to tweets as a way to
associate it with an event or a topic) were analyzed
in a large-scale study of the spatio-temporal
dynamics of {\em memes}. In this work a hashtag was represented
as a tuple consisting of the coordinates of the hashtag's location over time.
They used a simple model to find interesting insights about
the adoption and spread of memes in social media. Memes are information which
emerges from social networks and spreads in a viral way. However, meme
dissemination does not necessarily resemble how other types of information
will propagate, such as information about events that originate outside of the network (i.e., exogenous
events). Following this motivation, Kalyanam et al.~\cite{10.1371/journal.pone.0166694}
studied how exogenous events, in this case real-world news, propagate in social media.
In their work, they modeled news events based on the interarrival time between
social media posts, without considering any of the geographical information
associated to the event. Their goal was to model the intensity of the user activity
that is triggered by a real-world news event. Though, in our current work we also study real-world news using the
same data extraction technique as Kalyanam et al.~\cite{10.1371/journal.pone.0166694},
our approach differs in that our event model is not based on the interarrival
times of tweets, but rather on the geographical context of social
media information.

In a different type of study, Leetaru~\cite{leetaru2011culturomics}
performed a large-scale
analysis of 30
years of digitized news articles.  The author computed sentiment scores and
geolocation for each article.  The study indicated that some critical events
in the past, such as social revolutions, could have been forecasted by
looking at sentiment scores over time.  In addition, the author performed
community detection on country graphs by analyzing news in
which two or more countries were involved.  In this sense, our approach is
similar, because we model countries in terms of their co-occurrence in news.
However, our work is focused on automatic information extraction from online
social streams and on the creation of a more general representation.  We do
not focus on the analysis of sentiment of edited content from
formal news media outlets, but
on the interactions between locations, based on the aggregated reactions and opinions
of users of social platforms.

There are other approaches for event information modeling, which come from
the area of automatic text summarization. Chakrabarti and Punera~\cite{ICWSM112885},
for example, used hidden Markov models to represent sub-events, within a
broader event that is described using Twitter data. This model identified
sub-events based on the burstiness of the input data stream and the word
distribution of the main event. Another approach was presented
by Quezada and Poblete~\cite{quezada2013understanding}, which focused on automatic
summarization of multimedia content by using social media posts as surrogate
text for multimedia documents. A similar approach was used by Alonso et al.~\cite{Alonso:2015:WCW:2740908.2745397},
which was based on the \emph{social signature} of documents (that is, the set of keywords
of social media messages that point to a document), to augment the
document information.

Several other features that have been used for modeling
events on social media are worth mentioning, such as, users
involved in an event~\cite{Wang:LeadLine:2012}, the credibility of the information
that is published~\cite{Castillo:Information:2011}, and latent sentiment of
content~\cite{Wang:2012:SRT:2390470.2390490}, among others.
In addition, temporal features for events have been used in tasks such as,
the detection of events based on the temporal dynamics of their mentions
in social media~\cite{DBLP:journals/corr/GuilleF15}, and
also for event categorization~\cite{Ritter:2012:ODE:2339530.2339704}.
Nevertheless, we do not use those features at this time.

Certain studies focused specifically on the task of detecting events and
tagging their relevant geolocations.  In particular, some works targeted the
detection of localized
events~\cite{Watanabe:Jasmine:2011,Abdelhaq:EvenTweet:2013,Walther:2013fb,Lee:A:2011,Krumm:2015},
others the detection of global events~\cite{sankaranarayanan:twitterstand:2009},
and the detection of critical
events~\cite{Sakaki:Tweet:2013,DeLongueville:2009}.  Dong et
al.~\cite{Dong2015}, specifically, considered that events had different
temporal and spatial scales and proposed a multi-scale event detection
approach for social media.  This approach focuses on detecting and
reporting events with geolocalization.  Our current approach differs from
existing work, in that we create an aggregated representation of
the information about real-world events, producing a high-level
representation that includes the event's geographical
context, which is extracted from social media.
In addition, we enrich the information about an event by
using the locations of the users that post information about it.
%
%In the work of Kamath et al.~\cite{Kamath:Spatio-temporal:2013},
%already mentioned, the authors examine the spatio-temporal dynamics of
%hashtags of geo-tagged tweets. 
%Their work determines three categories for geographical movement;
%local, regional and worldwide, and three for peak analysis; slow, medium
%and fast.
%The authors conclude that physical and geographical distances are
%a constraint to the movement and adoption of hashtags.
%However, a hashtag might not represent a real-world event.
%One problem of these approaches is that they focus
%in one aspect of the data: either
%the text generated by users or the geographical distribution
%of those users, and do not allow to make in depth analysis
%of this two aspects.
%In addition, none of them allow to inspect relationships between
%countries as consequence of real world events.

Wang et al.~\cite{Wang:LeadLine:2012} visualized topics based on the
extraction of geographical entities from tweet text. They did not use this information to
establish the location of an event, but rather for event exploration.
SensePlace2~\cite{MacEachren:SensePlace2:2011} is a Visual Analytics
tool that allows users to explore a set of tweets and models them by
showing two geographical types of information: the locations from
where users discussed the topic and the locations being
mentioned in tweets. However, unlike our work, this information was only used
at single tweet level, and not at event level.

In the domain of cyber-physical systems, {\em events} are viewed as
conditions of interest~\cite{st-model_2009} within a
cyber-physical system, or as the co-occurrence of two people in the same
physical place~\cite{STEvent_2010}.  In general, events are modeled
according to the state of the objects in the system, considering attributes,
time and location.  The work presented by Tan et al.~\cite{st-model_2009} bears
certain similarities with our own, in the sense that they considered an event
to encompasses multiple information about a condition of interest in the system (in our
case in the online social network), including time and physical
locations. In addition, the authors defined different kinds of temporal and
geographical scopes for their events, which are similar to our definition of
{\em event impact}. The main difference relies in that our approach aims to
capture
high-level information of how a complex exogenous event, such as a news
event, is perceived by social network users in an aggregated way. Therefore,
we focus on geopolitical divisions as units of aggregated spatial
information and on representing geopolitical interactions.

Despite that the idea of adding spatio-temporal context to social media data
is not novel, to the best of our knowledge our work is the first that
formally introduces {\em protagonist} and {\em interested} locations in a high-level
event representation.  The novelty of our approach relies on the extension
of the notion of spatial context, first by associating real-world news to
one or more protagonist locations, and second by associating real-world news
to the locations where they generated interest.  In addition, our work does
not focus on event detection, classification or summarization, as most
of the prior work on event analysis does.


\section{Quantitative Analysis of News Events}
