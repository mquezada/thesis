\section{Introduction}
\label{sec:introduction}

%\mq{hablar de multimodal summarization, diferentes tamaños de topicos, clustering incremental, 
%redundancia ayuda a word embeddings?, complejidad de otros metodos}

%%%%% Index
% Context
% Motivation
% Problem statement
% Contributions

%%%% Content


% Context 
% \subsection{context}

%Un modelo de "representacion" de informacion (de evento en SM) de manera compacta 
%para incorporar nuevo conocimiento.... con aplicaciones A, B y C

%Una forma de modelar la informacion .... (alto nivel)
%Describir semi-formalmente el modelo 
%Poder separar "señales" o "aspectos" del evento 

%Presentar casos de estudio que muestren las N aplicaciones. Mostrar ejemplos de 
%cada aplicacion.

%ej, aplicacion "event retrieval": dado un evento, identificar contenido relevante, topicos o aspectos
%relevantes al evento real (?) 

%modelo con WE exacerba diferencias en los documentos para separar topicos usando UCG (redundancia)

%hipotesis con respecto a agrupar por url: contenido relevante es mas probable de ser compartido con las mismas
%urls que contenido no relevante 


% Nowadays, social media platforms have become a main source of news. According to
% a recent survey, by 2018 about two thirds of US adults used social media as a
% source of news~\cite{pewresearch}, and about 70 percent of Twitter users get
% news from there. With their scale, popularity, and ease of use, social media
% platforms have drastically lowered the entry barriers to disseminate newsworthy
% information in real time. 
%Due to that, news events are often reported earlier in
%social media than in traditional news media. 

% In particular, impactful or breaking real world events (such as the 2015 Nepal
% Earthquake or the 2017 Brexit) overflow social media with millions of messages.
% For that reason, it is a hard task for humans to go through every message in
% order to understand what an event is about. 


The sheer amount of newsworthy information published by users in social media
platforms makes it necessary to have efficient and effective methods to filter
and organize content. 
%
In this scenario, off-the-shelf methods fail to process large amounts of data,
which is usually approached by adding more computational resources. 
%
Simple data aggregations can help to cope with space and time constraints,
while at the same time improve the effectiveness of certain applications, such
as topic detection or summarization. 


The overload of information in social media streams makes it difficult for users
to obtain a comprehensive account of different emerging news events to get a 
complete picture of an event or to find relevant messages about it.
%
The ability to identify and summarize relevant information about newsworthy
topics that are accounted for by users in social media can be of great use to
society.
%
This is particularly true in the case of high impact and breaking news, like
natural disasters and crisis situations~\cite{ICWSM1817816}.
%\bp{agregar mas citas}.
%
In general, social media platforms present their content to the consumer in one
of two ways: (1) by displaying user posts in reverse chronological order, or (2)
by displaying posts in a personalized fashion, showing first what they deem more
interesting to the user.
%
However, if the user consuming social media content 
is seeking information about a specific event, the aforementioned approaches can
result also in unnecessary exposure to redundant, irrelevant, or even misleading
information.
%
In this context, the delivery of key pieces of information to the user is important,
and even critical in some cases; 
%
for instance, social media is often used as a complementary source of
information for disaster management and
response~\cite{ICWSM1817816,Sarmiento:2018:DDE:3201064.3201077}
%\bp{agregar citas}
and also to obtain information about newsworthy events not yet reported by
formal news outlets. 
%
In this sense, this need to produce high-quality real-time information requires 
effective ways to filter and organize information about real-world events on social media.
%

Despite the usefulness of social media as a worldwide news information source,
its consumption is not without challenges.
%
These challenges include, among many others, correctly assessing information
veracity and relevance to a specific topic, as well as dealing with a huge
volume of data with variable quality. 
%
In this sense, the consumption of social media content for information seeking 
about particular events requires effective ways to filter and organize streaming data.
%%
%
%Specifically, the task of organizing and filtering
%
%Filtering and organizing information in social media can be a
%particularly challenging task.
%
In particular, as tweets can include links to images, videos, to
external web-pages or to other tweets, this makes their content multimodal.

%
%From the perspective of understanding current events, tweets contain a large volume of irrelevant and redundant content, which can often be misspelled, or unreliably capitalized, 
%
%On the other hand, we observe several 
The URLs shared in Twitter often link to varied types of media (articles,
images, or videos), which have been found useful for identifying conversation
topics~\cite{mishne2012twanchor}.
%
In the context of news events, users tend to quickly re-share information if the
event sparks high interest (see Section~\ref{subsec:info_forwarding}), which
results in many (near-)duplicate messages.
%
Dealing with content of these characteristics requires different approaches
than with traditional media~\cite{Alonso:2017:WHH:3091478.3091484}.
%
Prior work has acknowledged the need of techniques for dealing with this type of
data in this sense.

%%

In this chapter we focus on the problem of producing summarized representations of
social media data related to news events without significant loss of
information.
%
We present a compact or lightweight representation for social media content related to news
events.
%
This representation allows us to preserve information about the topics involved
in an event, and to identify relevant content, while reducing the volume of
data.
%
This is done by leveraging microblog features, such as redundancy
and re-sharing capabilities, by using surrogate texts from shared URLs and 
word embeddings for vector representation of documents.
%
This can be especially useful for online data processing about developing news
and news information seeking tasks.
%
To achieve this, our model annotates shared URLs with the text of the messages
in which they appear.
%
Here we use messages as {\em anchor text} or {\em surrogate text}. 
%
We first leverage this idea by identifying relevant URLs that are shared in
social media during a news event.
%
We discard URLs that are too general with respect to an event (e.g. a general
report), or generic (e.g. linking to the homepage of a news outlet) as we deemed
them as non-relevant.
%
Then we aggregate all the anchor texts associated with relevant URLs into {\em
documents}, and all the conversations around those documents as well. 
%
This representation allows us to achieve comparable clustering results to 
those obtained by using the complete data, while reducing running time and 
required memory.
%
This is useful when dealing with noisy and raw user-generated social media data.



As a preliminary way to study the usefulness of our event representation, we
applied it to three news events as a case study.
%
We observed that our representation reduces the amount of records needed to
describe an event by one order of magnitude, compared to using the raw tweets. 
%
We were also able to identify sub-topics by clustering the documents using dense
word representations, such as neural network--based word embeddings. 
%
We show that the clustering based sub-topic detection task using our proposed
representation yields similar external quality metrics to those obtained using
the entire data, 
%
hence, indicating that we are preserving relevant aspects of the information
while considerably reducing running time.

\paragraph{Contributions.} Our contributions in this chapter are the following:

\begin{enumerate}
  \item We present a novel compact representation for news event information in
  microblogs, based on shared URLs. This representation reduces data volume.

  \item We study the usefulness of our representation through three case
  studies. We do so by introducing a methodology to represent news events in
  Twitter and find sub-topics. We show that our approach displays comparable
  performance to baseline methods at a fraction of the computational resources.
\end{enumerate}


% problemas con representar social media content
% Representing content generally involves representing its constituent terms in
% another space. For instance, using weighted term frequency vectors, graph
% models, or probabilistic models over the distribution of terms. In social media,
% these methods result in highly dimensional sparse features, which are difficult
% to manage without large quantities of well-curated messages. On the other hand,
% it is not trivial to manage large scale data. Current methods can manage large
% volumes of messages at great computational expense or by splitting the dataset
% in smaller sets and dealing with them separately. For example, traditional ways
% of representation of text, such as term frequency weighting, involves the
% generation of thousands of dimensions on messages that are about 10 to 20 words
% long.

% Nonetheless, in the context of news events, user-generated content can be leveraged
% to estimate the relevance of the information and to annotate content from
% external sources. We start from the assumption that users share and comment on
% content that is relevant or interesting to them. Hence, the existence of several
% messages on a topic suggests that the topic is important to users. We can
% leverage this idea even futher: we can annotate the shared URLs in user posts
% with the same text that users put in the posts that contains those URLs. In the
% literature, the text that accompanies a URL is usually known as {\em anchor text}, or
%   {\em surrogate text}, and this idea has been extensively used in query log
% mining~\cite{Beeferman:2000:ACS:347090.347176,10.1007/978-3-540-30192-9_58}. We
% can also take advantage of the redundancy of posts in social media. When exposed
% to an emergent news, users tend to share information as quickly as
% possible~\cite{kalyanam2016prediction}, and this often results in duplicate and
% near-duplicate data. This information can be useful to assess the
% importance of certain aspects of the event. In this work, we exploit these clues
% in order to design a model to represent newsworthy information.

% reescribir
% We propose a lightweight representation of news events in Twitter by aggregating
% tweets based on the existence of shared URLs. Our representation groups tweets
% by common shared URLs into {\it components}, and also ignoring highly common
% URLs in the dataset, in order to avoid having large components of unrelated
% tweets. This representation reduces the amount of records to be processed by one
% order of magnitude, and we validate its capabilities by performing clustering on
% three selected news events to identify sub-topics. By using dense
% representations of words, such as neural network based word embeddings, our
% model generates a compact representation of news events. We show that we achieve
% similar performance on sub-topic detection at a fraction of the time and memory
% required. Our proposed representation allows us to process raw, un-curated
% Twitter data faster than with using individual tweets, and it is flexible enough
% to incorporate new data as the event evolves in time.

% We propose a simple model to represent information about news events in Twitter
% using aggregations of tweets that mention the same URLs, and that retweet or
% reply themselves. Then, we model the resulting sets of documents using dense
% vector representation of terms, such as neural word embeddings, using a corpus
% of 193 million news-related tweets. With this representation, we exploit
% features often discarded in similar works, such as repeated messages and URLs
% mentioned in tweets. This results in a smaller and more flexible representation,
% which allows us to perform some tasks more efficiently, and to incorporate new
% information to the model with little effort. Our event representation is
% compatible with current approaches regarding Twitter event data, such as event
% detection, filtering, and summarization. We show the effectiveness of this
% representation by characterizing topics in news events shared in Twitter, which
% can contain several repeated or irrelevant posts. The topics consist of diverse
% tweets that share URLs, which can be seen as a multimedia representation of
% topics, regardless of the content of these URLs (images, videos, news articles,
% etc.). 




% problemas: mucha info, redundante, etc. 
% explotamos eso para mejorar la origanizacion de la info
% mostramos que es util para: ...
% tiene estas caracteristicas: ...


% One particular problem is what is called information overload: when users are
% exposed to way more information than they can process and understand, and
% effective methods to find relevant information are missing, assuming that
% users know what they are looking for. In this context, automatic summarization
% was proposed as a way to tackle this problem. Given a document or a set of
% documents about a specific topic, the output of the summarization method is a
% gist that summarizes the content, under certain criteria and restrictions. 


% In this work, we approach the issue of information overload by modeling
% content associated with online news, as propagated on Twitter. 
% When a {\em breaking news} occur, users tend to share and
% forward information about it as quickly as possible {\bf REF PLOS}, and ...

% motivation
% \subsection{motivation} 
% user contributed content reflect opinions of users, not media
% importance given by users
% free annotations of urls 




% problems with social media
% \subsection{challenges}
% events are noisy (different levels of relevance)


% However, the scale and rate of publication makes it difficult to understand
% what is happening or what an event is about. Also, the presence of irrelevant
% posts makes it even more difficult to understand a specific event. Usually,
% when a user wants to know about a certain ocurrence, he or she performs a
% search using some keywords, and a chronologically ordered timeline of the most
% recent posts is then shown. In other cases, a special sorting is made in order
% to show the most relevant posts that specific user might like the most, or
% just the posts with the highest scores in terms of popularity. However, this
% is not enough to understand the different aspects or topics related to an
% event.

% Different approaches have been proposed to help users understand the content.
% For example, automatic text summarization is used in order to select the most
% salient phrases from a set of documents, aiming for optimizing diverse criteria,
% such as relevance, coverage, or popularity. However, nowadays online news media
% has increasingly shifted their focus towards multimedia content {insertar
% cita}. There have been some efforts in order to summarize content of different
% modalities, such as images or videos in isolation, but few consider a mixture of
% them. Our hypothesis is that a summary in different content modalities, such as
% images, videos, and text, is more effective than a text-only, or images-only
% summary in order to understand an event. In other words, our hypothesis is that
% a multimodal summary is more informative than a unimodal one. In this work, we
% discuss the challenges of the current approaches and tackle the problem of
% multimedia summarization of events in social media, by considering content
% independently of its nature.



% problem statement
% \subsection{problem statement}

% hypotheses
% \subsection{hypotheses}
% Hypotheses: 
% 1. grouping tweets by url improve the ...
% 2. the use of word embeddings (fasttext) improve the ...

% % contributions
% % \subsection{contributions}
% Our contributions are the following:
% \begin{enumerate}
%   \item We investigate 
%   \item
%   \item ?
% \end{enumerate}


% % outline of paper
% % \subsection{outline}
% This paper is organized as follows. Section ...

% % -----------------------------------------------------------------------

% % Twitter desc
% On Twitter, each post, called {\em tweet}, is a short text ---traditionally
% 140 characters long, but by the year of 2017 this limit was extended to 280
% characters---, which can contain arbitrary URLs, as well as {\em hashtags}
% (words prepended with a \#, serving to the tweet as tags), {\em cashtags}
% (prepended by a \$, serving as identifiers for stocks and shares), and mentions
% (names of Twitter users).