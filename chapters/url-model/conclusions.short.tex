\section{Chapter Summary and Conclusions}\label{sec:conclusions}

% %

% %

%%
% In a future work, we are interested in studying new strategies to represent
% external URLs using social media data, and how this can be used to better
% understand real-world events.


In this chapter, we presented a lightweight representation to embody newsworthy
information in social media, and a methodology to generate a representation from
a set of social media messages. 
%
Our representation leverages the use of shared URLs in the context of news
events to aggregate common information across messages, and allows us to
identify sub-topics in an event much faster than traditional or off-the-self
methods.
%
We observed that our representation is able to preserve the topical information
of an event.
%
At the same time, it is significantly smaller than the original
dataset, requiring much less computational resources to perform standard tasks.
%
This representation requires little data preprocessing, meaning that it is
convenient to use in a scenario having large, raw, un-curated data.
%
However, there is room for improvement in terms of the results.
%
For this, we need to perform a large scale evaluation, although it is hard to
find ground truth data for large, raw, un-curated social media
messages~\cite{Alonso:2015:WCW:2740908.2745397}.
%
For this case scenario, our methodology aims to help processing large quantities
of noisy, un-curated data around news events more effectively.
%
In future work we are interested in studying how to formally define and identify
topic-specific URLs in the events, which we believe are key to creating a useful
representation.
%
%We are also interested in looking for different strategies to represent
%topics in news events by aggregating content using external URLs on social media.

Overall, the proposed representation allows us not only to perform tasks such as
clustering more quickly, but also this would allow us to explore the URLs shared
in a certain event, or across several of them.
%
This points to the goal of performing high-level analysis of events, using
representations based on aggregated features on event data from social media.
%
The case study of topic clustering presented in this chapter aims to show the
capabilities of our representation to perform high-level analysis based on
aggregation of content based on shared URLs in news events.