\section{Discussion and Future Work}\label{sec:conclusions}

% % In this paper, we presented a lightweight representation to represent newsworthy
% information in social media, and a methodology to generate a representation from
% a set of social media messages.
% %
% Our representation leverages the use of shared URLs in the context of news
% events to aggregate common information across messages.
% %
% We observed that our representation is capable to preserve topical information
% of an event.
% %
% At the same time, it is suitable to be significantly smaller than the original
% dataset, requiring much less computational resources to perform standard tasks.
% %
% This representation requires little data preprocessing, meaning that is
% convenient to use in an scenario of having large, raw, uncurated data.
%%
% In a future work, we are interested in studying new strategies to represent
% external URLs using social media data, and how this can be used to better
% understand real-world events.

Our proposed representation allows us to identify sub-topics in an event much
faster than traditional or off-the-self methods.
%
However, there is room for improvement in terms of the results.
%
For this, we need to perform a large scale evaluation, although it is hard to
find ground truth data for large, raw, uncurated social media
messages~\cite{Alonso:2015:WCW:2740908.2745397}.
%
For this case scenario, our methodology aims to help processing large quantities
of noisy, uncurated data around news events more effectively.
%
In a future work, we are interested in studying new strategies to represent
external URLs using social media data, and how this can be used to better
understand real-world events.
%
For example, to study how these signals, such as URL sharing in social media,
can give us clues about the development of critical events, such as natural
disasters or breaking news.
