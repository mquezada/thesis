\section{The VQ-Event Model}
 
\todo[inline]{mover esto a chap 4}
We define an {\bf event} as a conglomerate of information that encompasses all
of the social media content related to a real-world news occurrence. 
%
Using this specification, which considers an event as a complex unit of
information, we study the type of collective reaction produced by the event on
the social network. 
%
In particular, we analyze the intensity or immediacy of the
social network's response. 
%
By analyzing the levels of intensity in activity induced by different exogenous
events to the network, we are implicitly studying the priority that has been
collectively assigned to the event by groups of independent individuals
\cite{barabasi2005origin, karsai2012universal}. 

We characterize an event's discrete activity dynamics by using
\emph{inter-arrival times} between consecutive social media messages within an
event (e.g., $d_i = t_{i+1}-t_i$, where $d_i$ denotes the inter-arrival time
between two consecutive social media messages $i$ and $i+1$ that arrived in
moments $t_i$ and $t_{i+1}$, respectively).


% We introduce a novel vectorial representation based on a {\em vector
% quantization of the interarrival time distribution}, which we call {\em
% ``VQ-event model''}. 
% %
We introduce a novel vectorial representation based on a vector quantization of
the inter-arrival time distribution, which we call ``VQ-event model". 
%
This model is designed to filter events based on the distribution of the
inter-arrival times between consecutive messages.  
%
The most representative inter-arrival times are learned from a large training
corpus. 
%
This approach is inspired by the {\em code-book-based representation} from the
field of multimedia content analysis, which has been used in audio processing
and computer vision~\cite{ff,Vaizman}.
%
In our proposed approach, our method learns a set of the most representative
inter-arrival times from a large training corpus of events; 
%
each one of the representative inter-arrival times is known as a {\em codeword}
and the complete learned set is known as the {\em code-book}~\cite{Vaizman}. 
%
Each event is then modeled using a vector quantization (VQ) that converts the
inter-arrival times of an event into a discrete set of values, each value
corresponding to the closest codeword in the code-book. 
%
The resulting VQ-event model is then a vector in which each dimension contains
the percentage of inter-arrival times of the event that were assigned a
particular codeword in the code-book.

%%

The VQ-event representation is relative to an event's overall size since the
model is normalized with respect to the number of messages in the event.
%
Therefore the only criteria that are considered in the model are the
inter-arrival times of each particular event. 
%
This model allows us to group events based on the {\em similarity of the
distribution} of their inter-arrival times. 
%
In those terms, we consider as high-activity events those events for which the
distribution of inter-arrival times is most heavily skewed towards the smallest
possible interval, zero. 
%
In other words, events for which the overall activity is extremely intense in
comparison with other events.

%%

We represent an event $\mathit{e}$, belonging to a collection of events
$\mathcal{E}$, as a tuple $(\mathcal{K}_\mathit{e}, \mathcal{M}_\mathit{e})$,
where $\mathcal{K}_\mathit{e}$ is a set of \emph{keywords} and
$\mathcal{M}_\mathit{e}$ is a set of \emph{social media messages}.
%
Both the keywords and the messages are related to a real-world occurrence. 
%
The keywords are extracted in order to succinctly describe the occurrence, and
the messages are posts from users about the event.

%%\mathit{i}

To learn the most representative inter-arrival times we perform the following:
%
for each $\mathit{e} \in \mathcal{E}$ with messages $\mathcal{M}_\mathit{e} =
\lbrack m_{1}^\mathit{e}, m_{2}^\mathit{e}, \ldots m_{n}^\mathit{e} \rbrack$ and
their corresponding time-stamps $\lbrack t_{1}^\mathit{e}, t_{2}^\mathit{e},
\ldots t_{n}^\mathit{e} \rbrack$ where $t_{\mathit{i}} \leq t_{\mathit{i}+1}
\forall \mathit{i} \in [1,n]$, we compute all the inter-arrival times
$d_{\mathit{i}}^\mathit{e} =
t_{\mathit{i}}^\mathit{e}-t_{\mathit{i}-1}^\mathit{e}$ (the value of $t_{0}$ is
considered equal to $t_{1}$ for initialization purposes). 
%
Then, the values of $d_{\mathit{i}}^\mathit{e}$ for all events in $\mathcal{E}$
are clustered to identify the {\em most representative} inter-arrival times.

%%

Once the most representative inter-arrival times have been learned, the vector
quantization for each event is produced as follows: 
%
for each event, obtain all the inter-arrival times, and quantize each of the
inter-arrival times to the closest codeword in the code-book.  
%
This process is summarized in Algorithm~\ref{alg:learn_representation}.
%
Line~\ref{alg:line:all_time_diff} collects all of the inter-arrival times for all
the events in $\mathcal{E}$ in \textbf{f}. 
%
Line~\ref{alg:line:cluster} is a clustering algorithm which takes \textbf{f} and
the number of clusters $k$ as inputs and returns the centroids of the clusters
as the output in \textbf{c}. 
%
The centroids can be thought of as the most representative inter-arrival times
for the event set $\mathcal{E}$. 
%
After that, the inter-arrival times of each event $e$ is vector quantized in
terms of the centroids to obtain a $k$-dimensional real valued representation of
the event (Line~\ref{alg:line:vq}). 
%
In this representation, each entry is the percentage of messages with that
particular codeword as the inter-arrival time.

\begin{algorithm}
  \caption{{\scshape Learn-VQ-Representation}$(\mathcal{E}, k)$}
  \label{alg:learn_representation}
  \begin{algorithmic}[1]
    \STATE $\textbf{f} \gets \{d_{i}^\mathit{e}\text{ } | \text{ } m_{\mathit{i}}^\mathit{e} \in
    \mathcal{M}_\mathit{e}, \mathit{e} \in \mathcal{E} \}$ \label{alg:line:all_time_diff}
    \STATE \textbf{c} $\gets$ {\tt cluster(}$\textbf{f}, k${\tt)} \label{alg:line:cluster}
    \FOR{$\mathit{e} \in \mathcal{E}$} 
      \STATE \textbf{\textit{e}} $\gets$ {\tt vq(}$d_{i}^\mathit{e}, \textbf{c}${\tt)}\\ \label{alg:line:vq}
    \ENDFOR
    \RETURN A representation in $\mathbb{R}^{k}$ of each event $\mathit{e} = (\mathcal{K}_\mathit{e}, \mathcal{M}_e) \in \mathcal{E}$.
  \end{algorithmic}
\end{algorithm}

%\mathit{e}
% \todo[inline]{uniforme con algoritmo en Background}
% \begin{algorithm}
%   \caption{{\tt learn\_representation()}}
%   \label{alg:learn_representation}
%   \begin{algorithmic}[1]
%     \REQUIRE Event set $\mathcal{E}$, and number of codewords $k$ in the codebook.\\
%     \ENSURE A representation in $\mathbb{R}^{k}$ of each event $\mathit{e} = (\mathcal{K}_\mathit{e}, \mathcal{M}_e) \in \mathcal{E}$.\\
%     \STATE $\textbf{f} \leftarrow \{d_{i}^\mathit{e}| m_{\mathit{i}}^\mathit{e} \in
%     \mathcal{M}_\mathit{e}, \mathit{e} \in \mathcal{E} \}$
%     \\ \label{alg:line:all_time_diff} \STATE \textbf{c} $\leftarrow$
%     {\tt cluster(}$\textbf{f}, k${\tt)} \label{alg:line:cluster}
%     \FOR{$\mathit{e} \in \mathcal{E}$} \STATE \textbf{\textit{e}} $\leftarrow$ {\tt
%       vq(}$d_{i}^\mathit{e}, \textbf{c}${\tt)}\\ \label{alg:line:vq}
%     \ENDFOR
%   \end{algorithmic}
% \end{algorithm}

%%

To illustrate events with different levels of intensity in activity we present
two examples taken from our analysis of Twitter data. 
%
These examples show the inter-arrival time histograms for the entire life-cycle of
the two events. 
%
In the first example, the majority of the messages about the death of political
leader Nelson Mandela (Figure~\ref{fig:hi:example-mandela}) arrive within almost
zero seconds of each other. 
%
On the contrary, the messages about The Oscars
(Figure~\ref{fig:hi:example-oscars}) are much more spread out in time.

%%

We note that, by using inter-arrival times to describe the intensity of the
activity of an event, we make our analysis independent of the particular
evolution of each event. 
%
By doing this, we put no restrictions on how high-activity events unfold in
time, for example, they could be: 
%
(a) events that start out slowly and suddenly gain momentum, 
%
(b) events that go viral soon after they appear on social media and then decay
in intensity over a long (or short) period of time, 
%
(c) events that from the beginning produce large amounts of interest and sustain
that interest throughout their long (or short) lifespan, or 
%
(d) events that are a concatenation of any of the above, etc.
