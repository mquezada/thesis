\section*{Results and Discussion}
%\newtext{ 
Our main objective in this work is to analyze the
  characteristics of high-impact events which differentiate them from
  other types of events. In particular, we identify how early in an
  event's evolution is it possible to determine if an event is going
  to be of high impact in the on-line social network.

Tables~\ref{table:high-impact-sample}
and~\ref{table:low-impact-sample} show examples of events from the high-impact category and 
low-impact category. We recall that the high-impact
events are those which were in the top 8\% of the ranking obtained by
sorting the event clusters according concentration of social media
posts in bins that correspond to the shortest time interval.  Table
\ref{table:high-impact-sample}, shows two events of different sizes
(large and small) and different scopes (one global and the other more local)
categorized as high impact in our dataset. The first event, the death
of Nelson Mandela, is one of the large events in the dataset, with
$\approx$ 134,000 tweets. The histogram representation of this event,
shown in Figure \ref{fig:nelson_mandela} suggested that more than $80\%$ of
the tweets arrived almost instantaneously. This is an event of
international political and social importance, which created an
overwhelming flood of messages on social media.  Hence, it makes sense
for such an example to be placed in the high-impact category.  The
second event, on the other hand, about the 2013 Mumbai Gang Rape is of
much smaller scale, with a total of $\approx$ 1,700 tweets.  However,
this event caused considerable amount of immediate reaction on social
media, with more than $48\%$ of its messages arriving almost
instantaneously.  Despite its smaller size, and its possible
confinement to a localized on-line community, this event has been
placed in the high-impact category.

Table \ref{table:low-impact-sample} shows events that have been
classified by our methodology in the category of low impact.  The
first event, about a teen surviving after hiding in the wheel of a
airplane, had only a little more than $25\%$ of its messages arriving
instantaneously although it had over 18,000 messages.  The second
event, about the damages caused by a tornado in Canada, did not garner
much immediacy in attention since only $7\%$ of its messages arrived
instantaneously. Most of the messages of this event were spaced out in
time. Even though, we cannot say whether or not this event had
significant implications in the real-world, we can say that it did not 
have considerable impact on the Twitter network. The lack of interest
could be due to several factors that are currently beyond the scope of
this work, ranging from the lack of Twitter users in the locality of
the event, to the fact that the tornado did not cause big damages. 

Fig.~\ref{fig:highest}, Fig.~\ref{fig:low}, Fig.~\ref{fig:cdf-highest} and
Fig.~\ref{fig:cdf-lowest} show the average histograms and the average
cumulative distribution functions of the events corresponding to the
high and low impact categories, respectively.  Visually, the average
high-impact event vector representations starkly differs from that of
a low-impact event in that the histogram in Fig.~\ref{fig:highest}
seems to possess an exponential decay, while the histogram in
Fig.~\ref{fig:low} does not.  To test this hypothesis, we fit
exponential function of the form $f(x)=ae^{bx}$ to the event
histograms. Table~\ref{tab:curve_fitting} summarizes the results from
statistical significance tests performed on the parameters $a$ and
$b$, and on the residual least squares error used for fitting the
exponential curves.  The differences between these values is
statistically significant ($p$-value $\leq2.2\times 10^{-16}$), thus demonstrating
that high-impact events, on an average, fit the exponential decay
curve much better than their low-impact counterparts.  In addition,
Fig.~\ref{fig:param_est} shows two scatter plots with the resulting
exponential parameters $a$ and $b$.  We observe that the majority
($97.4\%$) of high-impact events have an exponent $b \leq -50$,
separating them unequivocally from other events.
%}


 

\begin{figure}
  \centering 
  \begin{subfigure}{\textwidth}
    %\includegraphics[width=\textwidth]{figures/avg_hist_3_7}
    \caption{High-impact event summary histogram. %\inote{Mauricio:
                                %change label for the   x-axis}
    }
    \label{fig:highest}
  \end{subfigure}%

  ~%add desired spacing between images, e. g. ~, \quad, \qquad, \hfill etc.
  % (or a blank line to force the subfigure onto a new line)
  \begin{subfigure}{\textwidth}
    %\includegraphics[width=\textwidth]{figures/avg_hist_0_1}
    \caption{Low-impact event summary histogram. %\inote{Mauricio:
                                %change label for   x-axis}
    }
    \label{fig:low}
  \end{subfigure}%
  ~ %add desired spacing between images, e. g. ~, \quad, \qquad, \hfill etc.
  % (or a blank line to force the subfigure onto a new line)

  \caption{\textbf{Fig.~\ref{fig:highest} and~\ref{fig:low} show the average
    histogram of the high-impact and lowest impact groups respectively in our dataset. The $y$-axis is in square root
      scale.
      % \inote{change labels of x and y axis}
    }}\label{fig:histograms}
\end{figure}
\begin{figure}
  \centering
  \begin{subfigure}{.5\textwidth}
    %\includegraphics[width=\textwidth]{figures/cdf-highest}
    \caption{Average cumulative distribution function of high-impact events. %\inote{Mauricio:
                                %change label for the   x-axis}
    }
    \label{fig:cdf-highest}
  \end{subfigure}%
  ~%add desired spacing between images, e. g. ~, \quad, \qquad, \hfill etc.
  % (or a blank line to force the subfigure onto a new line)
  \begin{subfigure}{.5\textwidth}
    %\includegraphics[width=\textwidth]{figures/cdf-lowest}
    \caption{Average cumulative distribution function of low-impact events. %\inote{Mauricio:
                                %change label for   x-axis}
    }
    \label{fig:cdf-lowest}
  \end{subfigure}%
  ~ %add desired spacing between images, e. g. ~, \quad, \qquad, \hfill etc.
  % (or a blank line to force the subfigure onto a new line)

  \caption{\textbf
      % \inote{change labels of x and y axis}
    }}\label{fig:cdfs}
\end{figure}


\begin{figure}
  \centering
  \begin{subfigure}{.5\textwidth}
    %\includegraphics[width=\textwidth]{figures/param_est_a}
    \caption{Coefficient $a$ parameter estimation summary per event impact category. %\inote{Mauricio:
                                %change label for the   x-axis}
    }
    \label{fig:param-a}
  \end{subfigure}%
  ~%add desired spacing between images, e. g. ~, \quad, \qquad, \hfill etc.
  % (or a blank line to force the subfigure onto a new line)
  \begin{subfigure}{.5\textwidth}
    %\includegraphics[width=\textwidth]{figures/param_est_b}
    \caption{Exponent $b$ parameter estimation summary per event impact category. %\inote{Mauricio:
                                %change label for   x-axis}
    }
    \label{fig:param-b}
  \end{subfigure}%
  ~ %add desired spacing between images, e. g. ~, \quad, \qquad, \hfill etc.
  % (or a blank line to force the subfigure onto a new line)

  \caption{\textbf{Fig.~\ref{fig:param-a} and~\ref{fig:param-b} show the parameter value estimation for $f(x)=ae^{bx}$.}}
  \label{fig:param_est}
\end{figure}


\begin{table}
  \centering
  \begin{tabularx}{\textwidth}{cccc}
    \toprule
    \textbf{Parameter} & \textbf{Mean (high-impact)} & \textbf{Mean (low-impact)} & \textbf{$p$-value} \\ \midrule
    $a$ & $0.7129$ & $0.0822$ & $\leq2.2\times 10^{-16}$  \\ 
    $b$ & $87.1903$ & $3.8013$ &$\leq2.2\times 10^{-16}$ \\
    Error & $0.003$ & $0.008$ & $\leq2.2\times 10^{-16}$ \\ \bottomrule
  \end{tabularx}
  \caption{\textbf
    }}
  \label{tab:curve_fitting}
\end{table}

Further analysis of the high-impact events shows significant
differences to other events, in the following aspects: (i) how the
information about these events is propagated, (ii) the characteristics
of the conversations that they generate, and (iii) how focused users
are on the news topic. In detail, high-impact events have a higher
fraction of {\em retweets} (or shares) relative to their overall
message volume. On average, a tweet from a high-impact event is
retweeted 2.36 times more than a tweet from a low impact event. The
most retweeted message in high-impact events is retweeted 7 times more
than the most retweeted message in a medium or low impact event. We
find that a small set of initial social media posts are propagated
quickly and extensively through the network without any rephrasing by
the user (plain forwarding). Intuitively, this seems justified given
general topic urgency of high-impact events. Events that are not
high-impact did not exhibit these characteristics.

Our research also revealed that high-impact events tend to spark more conversation
between users, 33.4\% more than other events. This is reflected in the
number of {\em replies} to social media posts. The number of different
users that engage in high-impact events is 32.7\% higher than in
events that are not high-impact. Posts about high-impact events are
much more topic focused than in other events. The vocabulary of unique
words as well as {\em hashtags} used in high-impact events is much
more narrow than for other events. Medium and low impact events have
over 7 times more unique hashtags than high-impact events. This is
intuitive, given that if a news item is sensational, people will
seldom deviate from the main conversation topic.

% We have presented an analysis of high-impact news events based on
% the data of their entire life-cycle in the social network. We used
% the arrival time intervals to create a model that allows us to
% classify the event according to its impact. Nevertheless,

In a real-world scenario, in order to predict if an early breaking
news story will have a considerable impact in the social network, we
will not have enough data to create its impact-based model, i.e., we
will not yet know the distribution of the speed at which the social
media posts will arrive for the event. For instance, an event can
start slowly and later produce an explosive reaction, or start
explosively and decay quickly to an overall slower message arrival
rate. Still, reliable early prediction of very high-impact news is
important in many aspects, from decisions of mass media information
coverage, to natural disaster management, brand and political image
monitoring, and so on.

For the task of early prediction of high-impact events, we use features 
that are independent of our impact-based model, such as
the retweets, the sentiment of the posts about the event, etc. 
These features are computed on the early 5\% of messages about the event.
% More details are provided in the supplementary material.
The results are an average from a 5-fold cross validation with
randomly selected 60\% training, 20\% validation and 20\% test splits.
The high-impact events are identified with a precision of 82\% using
only the earliest 5\% of the data of each event
(Table~\ref{tab:classification_results}).  Additionally, we were able to
identify with high accuracy a considerable percentage of all
high-impact events ($\approx 46\%$) at an early stage, with very few
false positives (Table~\ref{tab:classification_results} and~\ref{tab:confusion_matrix}).

\begin{table}
  %\begin{adjustwidth}{-10mm}{-10mm}
  \centering
  {\small
    \begin{tabularx}{\textwidth}{lcccc|cccc}
      \toprule
      & \multicolumn{4}{c}{\textbf{Early 5\% Tweets}} & \multicolumn{4}{c}{\textbf{All Tweets}} \\
      \midrule
      & FP-Rate & Precision & Recall & ROC-area & FP-Rate & Precision & Recall & ROC-area \\
      % \midrule
      high-impact & 0.009 & 0.819 & 0.455 & 0.900 & 0.01 & 0.830 & 0.540 & 0.945 \\
      non-high-impact & 0.545 & 0.954 & 0.991 & 0.900 &  0.460 & 0.960 & 0.990 & 0.945 \\
      \bottomrule
    \end{tabularx}
  }
  \caption{\textbf{Classification of high-impact events.}}
  \label{tab:classification_results}
  %\end{adjustwidth}
%                                                                                                                                448,1         93%
\end{table}

\begin{table}
  \centering
  % {\scriptsize
  \begin{tabularx}{\textwidth}{lcc|cc}
    \toprule
    \multirow{2}{*}{ }& \multicolumn{2}{c}{\textbf{Early 5\% Tweets}} & \multicolumn{2}{c}{\textbf{All Tweets}} \\
    \midrule
    % \cmidrule{2-5} \cline{2-5}
    & high-impact & non-high-impact & high-impact & non-high-impact \\
    % \midrule
    high-impact & $194$ & $232$ & $230$ & $196$\\
    non-high-impact & $43$ & 4,765 & 47 & 4,761 \\
    \bottomrule
  \end{tabularx}
  % }
  \caption{\textbf{Confusion matrix for high-impact events prediction.}}
  \label{tab:confusion_matrix}
\end{table}



The precision using only the early tweets is almost as good as using
all tweets in the event (0.819 to 0.830). This suggests that the
social network somehow acts as a natural filter in separating out the
high-impact events fairly early on.  The recall goes from 0.455 to
0.540. This indicates that there are some high-impact events which
require more data in order to determine what kind of impact they will
produce, or events for which impact occurs due to random conditions. A
detailed description of the features and different classification
settings are provided in the supplementary material.%\supplementary.