\section*{Abstract}

On-line social networks publish information about a high volume of
real-world events almost instantly, becoming a primary source for
breaking news.  Many of the events reported in social media can be of
high impact to society, such as important political decisions, natural
disasters and terrorist actions, but they might go unnoticed in their
early stages due to the overload of other information.  Is it possible
to clearly and quickly identify which of these news events are going
to have substantial impact before they actually become a trend in the
network?

We research 5,234 real-world news events encompassing 43 million
messages discussed on the Twitter microblogging service for
approximately 1 year.
% A novel vectorial representation for events
% based on the message arrival rates on social media is introduced.
% Such a representation enables an automatic separation of events
% into different categories based the chatter or \emph{impact} they create on
% social media.
%
We observe a consistent and strikingly different collective reaction
from users when exposed to high-impact events.  This reaction is
independent of an event's size and scope.  We further observe that
high-impact events have characteristics that are quite distinguishable
at the beginning stages of their outbreak.  This allows us to predict
with high precision the top 8\% of events that will have the most
impact on users by just using the first 5\% of the information of an
events' lifetime evolution.  This strongly implies that high-impact
news is naturally filtered by the social network, engaging users early
on, way before it is brought to the mainstream audience.


\section*{Introduction}
%\section{a}

% Motivation
Social media has become a primary source of breaking news information
for millions of users all over the world \cite{Kwak:2010}. On-line
social networks along with mobile internet devices have crowdsourced
the task of disseminating real-time information. As a result, both
news media and news consumers have become inundated with much more
information than they can process. To handle this data overload, it is
important to find ways to quickly filter information that has the
potential of creating an impact on society. Fast identification of
high-impact news events, at both global and local scales, can help
improve information delivery, as well as journalistic coverage and
emergency responses from authorities during crisis situations. We
address this challenge by analyzing the properties of real-world news
events in on-line social networks and present our main findings of how
high-impact events can be clearly identified in the early stages of
their outbreak.

% Brief background on the problem

The study of information propagation on the Web has sparked tremendous
interest in recent years. Current literature on the subject primarily
considers the process through which a {\em meme}, usually a piece of
media (like a video, an image, or a specific Web article), gains
popularity
%\cite{Castillo:2014,Szabo:2010,Lerman:2010,Tatar:2014,Pinto:2013,Ahmed:2013,Li:2016:concept:drift,
%  Liu:2015:UN}.  
\cite{Castillo:2014,Szabo:2010,Lerman:2010,Tatar2014,Pinto:2013,Ahmed:2013,Li:2016:concept:drift,
Liu:2015:UN}.  
Although relevant, this research faces two major
shortcomings when it comes to the identification of impactful
newsworthy events.  The first limitation is that a meme represents a
simple information unit and its propagation behavior does not
correspond necessarily to that of more complex information such as
news events. News events are usually diffused in the network in many
different formats, e.g., a particular news story such as an {\em
  earthquake in Japan} can be communicated through images, URLs,
tweets, videos, etc. The second limitation of prior work comes from
how the {\em impact} of information is measured. In scientific
research, impact has been traditionally related to the total amount of
attention received by a subject
\cite{berger2012makes,iribarren2011branching,guerini2011exploring,mills2012virality,gaugaz2012predicting}.
This is, content from on-line social networks that receives
votes/comments/shares above a certain threshold deemed {\em viral} or
{\em popular}, therefore having impact. One problem with this
definition is that it does not consider events of different sizes and
scopes, and that it is biased towards very large events. It is
reasonable to consider as high-impact a global breaking news event
with world-wide coverage that produces a high volume of social media
messages in a short time.  However, there are other types of events
that can produce a similar reaction in smaller on-line communities,
such as a particular country
% For instance, a global
% event with world-wide coverage is big news because of the volume of social media
% posts produced by on-line social networks over a short period of time.
% Therefore, it is reasonable to consider this news as high-impact.
% Nevertheless, there are events that create high-impact in smaller
% on-line communities, such as a particular country 
(e.g., the
withdrawal of the main right wing presidential candidate in Chile due
to psychiatric problems, just before
elections \cite{chile_elections}).
%\footnote{\url{http://www.telegraph.co.uk/news/worldnews/southamerica/chile/10188216/Chilean-presidential-candidate-pulls-out-of-election-with-depression.html}
%  (Accessed: August 25, 2015.)}). 
Clearly, events of local scope do not produce as much social media
activity as events of global scope, but they can create a strong and
immediate reaction from users in local networks, and therefore should
also be considered as high-impact \cite{ReisBOPKA15}. Conversely,
there are large events which do not produce strong reactions, such as
{\em The Oscars} (Fig.~\ref{fig:oscars}), which span a long
period of time and are discussed by social network users for weeks or
even months. When considered in its full, this event does generate a
great volume of social media messages, but this activity is
distributed over time without producing an immediate reaction from the
network's users. With these considerations in mind, it is evident that
a variety of aspects need to be considered when determining the level
of impact of a particular information topic. Furthermore, qualitative
evidence gathered from consulting a journalist, indicates that they
too face the issue of not having a unique measure of impact.

% when
% consulting journalists on how news media sources measure the impact of
% news, we learn that they too face the issue of not having a clear way
% to approach this problem.

% Our contributions

%\newtext{
Our work focuses on high-impact news events in social media
with the following contributions:
\begin{enumerate}
\item We introduce a new concise measure of information impact in social
media that is independent of the size and scope of the event, but
representative of the overall immediacy of the social network's reaction.
\item We determine the existence of unique characteristics that
differentiate how high-impact events propagate in the social network.
\item We show that an important portion of high-impact news events can be
predicted very early in their lifecycle, indicating that this information is
spontaneously identified and filtered early on by social network users.
\end{enumerate}