\section{Introduction}

Users publish a high volume of information about real-world events on social
media almost instantly.
%
This makes social media a primary source for breaking news.
%
Some of these real-world events can end up having a very strong impact on the
network.  
%
The effect of such events can be analyzed from several perspectives, one of them
being the intensity and characteristics of the collective activity that it
produces in the social platform. 
%
We research 5,234 real-world news events encompassing 43 million messages
discussed on the Twitter microblogging service for approximately 1 year, using a
portion of the dataset collected using the methodology described in
Chapter~\ref{chapter:data}.
%
We show empirically that exogenous news events naturally create collective
patterns of bursty behavior in combination with long periods of inactivity in
the network. 
%
This type of behavior agrees with other patterns previously observed in other
types of natural collective phenomena, as well as in individual human
communications. 
%
In addition, we propose a methodology to classify news events according to the
different levels of intensity in activity that they produce. 
%
In particular, we analyze the most highly active events and observe a consistent
and strikingly different collective reaction from users when they are exposed to
such events. 
%
This reaction is independent of an event's reach and scope.  
%
We further observe that extremely high-activity events have characteristics that
are quite distinguishable at the beginning stages of their outbreak.  
%
This allows us to predict with high precision, the top 8\% of events that will
have the most impact in the social network by just using the first 5\% of the
information of an event's lifetime evolution. 
%
This strongly implies that high-activity events are naturally prioritized
collectively by the social network, engaging users early on, way before they are
brought to the mainstream audience.


% Motivation

Social media is now a primary source of breaking news information for millions
of users all over the
world~\cite{Kwak:2010,petrovic2013can,broersma2013twitter,tandoc2016most,Rogers:2013:DTT:2464464.2464511}.
%
Social media along with mobile internet devices have crowd-sourced the
task of disseminating real-time information. 
%
As a result, both news media and news consumers have become inundated with much
more information than they can process. 
%
One possible way of handling this data overload is to find ways to filter and
prioritize information that has the potential of creating a strong collective
impact. 
%
Understanding and quickly identifying the type of reaction that certain
exogenous events will produce in social media, at both global and
local scales, can help in the understanding of collective human behavior.
%
This may as well as improve information delivery, journalistic coverage and
crisis management, among other things. 
%
We address this challenge by analyzing the properties of real-world news events
in social media, showing that they corroborate patterns previously identified in
other case studies of human communications. 
%
In addition, we present our main findings of how news events that produce
extremely high-activity can be clearly identified in the early stages of their
outbreak.

% when
% consulting journalists on how news media sources measure the impact of
% news, we learn that they too face the issue of not having a clear way
% to approach this problem.

% Our contributions

%\newtext{

\paragraph{Contributions.} Our work focuses on high-activity events in social
media produced by real-world news, with the following contributions:
\begin{enumerate}

\item We introduce a methodology for modeling and classifying events in social
media, based on the intensity of the activity that they produce. 
%
This methodology is independent of the size and scope of the event, and is an
indicator of the impact that the event information had on the social network.

\item We show empirically that real-world news events produce collective
patterns of bursty behavior in the social network, in combination with long periods of
inactivity. 
%
Furthermore, we identify events for which most of their activity is concentrated
into very high-activity periods, we call these events {\em high-activity
events}.

\item We determine the existence of unique characteristics that differentiate
how high-activity events propagate in the social network.

\item We show that an important portion of high-activity events can be predicted
very early in their life-cycle, indicating that this type of information is
spontaneously identified and filtered collectively, early on, by social network
users.

\end{enumerate}
%}

%We focus on high-impact news events in social media, contributing by (i)
%\textcolor{blue}{defining a new concise way for measuring information impact that
%is independent of the size (whether large or small) and scope (whether
%local or global) of the event, but is representative of the urgency
%and immediacy of the reaction of users on the social media} (ii)
%determining the existence of unique characteristics that differentiate
%how high-impact exogenous events are propagated in the network, and
%(iii) show, through the creation of an early prediction model for
%high-impact events, that these types of news events are naturally
%identified and filtered by the network at very early stages of their
%outbreak.