
\section*{Conclusion}

We show that there are several properties that separate how
high-impact news events evolve in Twitter in comparison to other
events. We have created a model for events that allows us to do
unambiguous classification of high-impact events based on their impact
in the social network, in terms of the distribution of their
inter-message arrival rates. This definition does not have some of the
problems that current notions of virality and popularity have. Some
characteristics of high-impact events are that they are forwarded more
often by users, and generate a greater amount of conversation than
other events.  Social media posts from high-impact news events are
much more focused on the news topic. Our experiments show that there
are several properties that can suggest early on if an event will have
high-impact on the on-line community.  We can predict a high number of
high-impact events {\em before} the network has shown any type of
explosive reaction to them. % Using simple off-the-shelf feature based
classifiers, we can
% predict many high-impact events with high precision.
This suggests that users are collectively quick at deciding whether an
event is important or not.  However, there does exist a fraction of
events which will become high impact, despite not presenting
patterns of other high impact events during their early stages.  These
events are likely to be affected by other factors, such as random
conditions found in the social network at the moment and require
further investigation.