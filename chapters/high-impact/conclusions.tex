\section{Chapter Summary and Conclusions}

In this chapter, we studied the characteristics of the activity that news events
produces in the Twitter social network. 
%
In particular, we proposed to measure the impact of the real-world news event on
the online social network by modeling the user activity related to the event
using the distribution of their inter-arrival times between consecutive
messages.
%
In our research we observed that the activity triggered by real-world news
events follows a similar pattern to that observed in other types of collective
reactions to events, 
%
namely by displaying periods of intense activity as well as long periods of
inactivity.
%
This model of user activity allowed us to perform high-level analysis of news
events, by characterizing them and identifying groups of similar user activity.


We extended this analysis by identifying groups of events that produce
a much greater concentration of high-activity than other events. 
%
We showed that there are several specific properties that distinguish how
high-activity events evolve in Twitter, when comparing them to other events. 
%
We designed a model for events, based on the code-book approach, that allows us to
do unambiguous classification of high-activity events based on the impact
displayed by the social network. 
%
This definition does not have some of the problems that current notions of
virality and popularity have. 
%


Some notable characteristics of high-activity events are that they are forwarded
more often by users, and generate a greater amount of conversation than other
events.
%
Social media posts from high-activity news events are much more focused on the
news topic. 
%


Our experiments showed that there are several properties that can suggest early on
if an event will have high-activity in the online community. 
%
We can predict a high number of high-activity events {\em before} the network
has shown any type of explosive reaction to them. 
%
Using simple off-the-shelf feature based classifiers, we can predict many
high-impact events with high precision. 
%
This suggests that users are collectively quick at deciding whether an event
should receive priority or not. 
%
However, there does exist a fraction of events which will create high activity,
despite not presenting patterns of other high activity events during their early
stages.
%
These events are likely to be affected by other factors, such as random
conditions found in the social network at the moment and require further
investigation.