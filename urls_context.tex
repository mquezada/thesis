\documentclass{article}

\usepackage{booktabs}
\usepackage{tabularx}

\title{Rough draft}

\date{\today}
\begin{document}

\maketitle

\section{Introduction}

\section{Possible questions}

\begin{enumerate}
\item How to leverage the information about shared URLs?
\item Is it possible to make use of the redundancy in text to extract
  useful knowledge?
\item Can we improve the outcome of topic detection methods using the
  existence of shared URLs?
\item The existence of an specific URL tells us something about the
  topic of the tweet?
\item Is there a difference between the content about an ongoing news
  event and non-relevant content, such as spam, by looking at the
  patterns of shared URLs?

\end{enumerate}

\section{So what}
% so what? what do we want to achieve?

By studying the patterns of shared URLs in news events in Twitter, we
would want to achieve the following:

\begin{itemize}
\item Be able to identify different aspects of an event by looking at
  sharing patterns of URLs
\item Differentiate relevant content (event-related content) from
  non-relevant content (non event-related)
\item Understand how an event develops in terms of the external
  content from news sources and other sources.
\item Understand patterns of shared content: news source-related vs
  non news source-related
\end{itemize}



\section{Contributions}

The contributions of this work are the following:

\begin{itemize}
\item We propose a lightweight representation of news events from
  social media based on the co-occurrence of URLs mentioned in the
  text. With this model, we can ``compress'' social media posts,
  leveraging the redundancy and noise from raw datasets.
  
\item We propose and implemented a system for labeling tweets related
  to news events, to assign multi-labels to tweets ranked by relevance. (?)

\item We provide case studies validating two applications of our
  model: exploring the usage of URLs about news events in Twitter, and
  identifying sub-topics in the events using our model, which allows
  us to process large amounts of tweets with standard techniques.
  
\end{itemize}







\section{Methodology}

Given an event $E=(T, U)$, of tweets and urls:

\begin{itemize}
\item For each tweet $t$ in $T$:
  \begin{itemize}
  \item Extract all occurences of URLs $u_i$
  \item Create pairs $(t, u_i)$ for each URL
  \item Create pairs $(t, rt)$ and $(t, rp)$, where $rt$ is a tweet
    such that $t$ retweets $rt$, and $rp$ is a tweet such that $t$
    replies to $rp$; do not create them if they do not exist
  \end{itemize}
\item Create a graph $G$ using the edges $(t, o)$, generated above,
  where $o$ is an URL or a tweet.
\item Return $G$
\end{itemize}

We call a connected component of $G$ a \emph{component}, denoted $c$,
identified by all URLs $u^c_1, \ldots, u^c_{|c|}$ that are nodes in
the connected component.

One issue with this approach is that it may produce very large
components. Take, for instance, two large components. If one tweet in
one component replies or retweets a tweet in the other component, or
it shares an URL already present in the other component, then the two
will merge. This can happen when sharing the same image across
different news articles, or when citing the root URL of a news source
(e.g. citing an article from BBC news media, and also citing the
root URL of BBC).

One way to reduce this effect is to choose only one URL in a tweet
that shares more than one. Also, by not considering replies as a way
to join two tweets. In this case, the methodology results in smaller
components, with less URLs in each one. One disadvantage of this
approach is that the same tweet (the same text) may end up in
different components.

The following procedure describes this version of the methodology:

Given an event $E = (T, U)$ of tweets and URLs:

\begin{itemize}
\item For each tweet $t$ in $T$:
  \begin{itemize}
  \item Extract all occurences of URLs $u_i$
  \item For each $u_i$, create a copy of $t$, called $t^i$, same with
    retweet $rt$ and reply $rp$
  \item Create pairs $(t^i, u_i)$, and $(t^i, rt^i)$ and $(t^i, rp^i)$
    if applicable
  \end{itemize}
\item Create a graph $G$ using the edges $(t, o)$ generated above,
  where $o$ is an URL or a tweet
\item Return $G$
\end{itemize}

In the following sections, we will use this version of the method.


\section{Topic detection}

One task we wanted to achieve is to identify topics of the event using
the aforementioned approach.

Our main hypothesis is that the tweets in a component refer to the
same topics. Reasons:

\begin{itemize}
\item When sharing a news article, users would often comment
  on the most striking aspect of the news.
\item When a news event breaks, only certain aspects will be known at
  the time, and news sources and users would focus only on those
  aspects.
\item 
\end{itemize}

In other words, we assume that the text in a tweet that share an URL
is a surrogate for the content of that URL, in the context of news
events.

\paragraph{Weaker(?) Hypothesis} The tweets of a component share similar
patterns? The components form clusters? \\

In order to test this hypothesis, we performed clustering on the
components and compared the resulting clusters with a simple baseline,
which is clustering the tweets as if they are not grouped.

To represent each component as a vector, we computed word embeddings
from the entire collection of events. We used the text of 193 million
tweets, with 1.1 million unique words, and 3 billion words in
total. We represented a component as a vector by taking all the words
from all the tweets inside a component, computed the vector of each
word, and taken the average of all word vectors. The same process for
the baseline: take all the words of a tweet, get the word embeddings
for each word, and average them.

\section{Case studies}

\subsection{Data and events}

We selected three events from our collection. Those events are a
terrorist attack in a hotel in Libya in 2015, the trial of Oscar
Pistorius in 2014 and 2015, and an earthquake in Nepal in 2015
(Table~\ref{tab:cs-events}).

\paragraph{Data preprocessing} For each tweet in each event, we
resolved the URLs contained in the text of the tweet, by following
redirects. Then we removed query strings and hashes from the resolved
URLs, with few exceptions (e.g., the query string \texttt{v} is used
by YouTube as the identifier of a video).

\begin{table}[]
\begin{tabularx}{\textwidth}{@{}Xrr@{}}
\toprule
Event name            & Tweets & Unique URLs \\ \midrule
Libya Hotel Attack    & $20\,077$  & $3\,412$        \\
Oscar Pistorius Trial & $68\,832$  & $9\,649$        \\
Nepal Earthquake      & $442\,221$ & $23\,116$       \\ \bottomrule
\end{tabularx}
\caption{Amount of tweets and URLs for the events used in the case studies.}
\label{tab:cs-events}
\end{table}


\subsection{Generation of ground truth}

In order to validate the usefulness of our model at identifying topics
in events, we asked a group of people to label some of the tweets from
the events of our case studies.

Due to the amount of tweets in each event, we sampled tweets...





\end{document}


