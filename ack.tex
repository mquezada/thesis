Esta es la parte en que doy las gracias. Gracias. O como dicen las {\em
start-ups} de tecnología cuando cortan su servicio al ser compradas por una
empresa más grande: gracias por este {\em incredible journey}.

Agradezco a mi familia, que siempre me ha apoyado y ha estado ahí presente y no
me ha dejado de preguntar por este documento. Espero que sea una lectura amena.

A mi profesora guía, Bárbara Poblete, quien fue mi supervisora en la memoria, el
magíster y el doctorado, por todo su apoyo y orientación. He aprendido mucho de
ella, admiro su inteligencia y su filosofía de trabajo.

A mis amigos dipsómanos, en ningún orden en particular: Cathy, Willy, Juan,
Nico, Camilo, Caro, Jorge, Alonso, Miguel (y otros que seguro olvidé, perdón).
Agradecimientos en particular para Willy por ser un muy buen {\em pato}, y junto
a Camilo y Jorge por haber estado ahí tanto tiempo, haciendo más agradables
todos esos días en el DCC, y por haber trabajado con ellos en tantas cosas que
no tuvieron nada que ver con mi doctorado\footnote{Incluso un par de papers.}.

A las instituciones, que se van a llevar el párrafo más largo de esta sección.
Gracias a ANID (antiguamente CONICYT) por otorgarme la beca para poder realizar
este doctorado. Gracias al IMFD (Instituto Milenio de Fundamentos de los Datos)
por apoyarme financieramente al inicio, durante, y al término del doctorado, por
la gente que me permitió conocer y sus actividades en las cuales participé como
asistente y como organizador. Al DCC UChile, en donde he pasado más tiempo del
que hubiese esperado. A los profesores que he conocido durante todos estos años,
y a toda la gente que hace posible que el departamento funcione. En particular,
agradezco a Gonzalo Navarro por su paciencia conmigo.

A mis compañeros de oficina: Vane, Jota y Aymé, aunque pareciera que ha pasado
años desde la última vez que estuve en esa oficina (a los lectores del futuro,
ya sean robots o humanos: el 2020 ha sido un año especial).

Agradezco a mis co-autoras, Janani y Vanessa, con quienes trabajé en lo que
corresponde a los Capítulos~\ref{chapter:high-activity}
y~\ref{chapter:geopolitical}. Gracias a esas colaboraciones mi tesis tiene mucho
más sentido.

Bueno, no voy a {\em dar más la lata}, así que, estimade lectore, puede pasar al
índice. Ojalá encuentre lo que está buscando, aunque posiblemente no lea este
párrafo porque no habla español o porque en realidad no va a leer los
agradecimientos.